\section{Główne funkcje}

\subsection{Metody klasy Parser}

\subsubsection{Unifikacja wejściowego łańcucha znaków}
\begin{lstlisting}
inline string Parser::UniformInputString(string s)
\end{lstlisting}

Zadaniem funkcji jest ujednolicenie wielomianu podanego na wejściu, w~taki sposób, by był jednoznaczny i~łatwy do dalszego przetworzenia. Jako argument przyjmuje ona pojedynczy obiekt klasy string. Po dokonaniu odpowiednich operacji, jako rezultat, zwraca również typ string. Użycie jej ma na celu ułatwienie przetwarzania w~kolejnym etapie, w~którym na podstawie podanego ciągu znaków, tworzony będzie obiekt wielomianu. Głównym zadaniem funkcji, jest weryfikacja, czy ciąg znaków podanych na wejściu reprezentuje poprawny składniowo wielomian. Sprawdzana jest liczba nawiasów otwierających i~zamykających oraz fakt, czy w~każdym miejscu wyrażenia liczba otwartych nawiasów jest nie mniejsza niż liczba nawiasów zamkniętych. Dodatkowo sprawdzane jest, czy na sąsiednich miejscach nie występują dwa operatory. Poprawne wyrażenie nigdy nie kończy się operatorem, a~zaczynać może się tylko minusem, literałem, cyfrą lub nawiasem otwierającym. W~czasie unifikacji wyrażenia, ignorowane są wszystkie występujące w~nim białe znaki. Warto zaznaczyć, że dopuszczalny jest tylko jeden znak, reprezentujący zmienną wielomianu. Nie ma ograniczeń co do wartości tego znaku, może być to znak 'a', 'x' lub jakikolwiek inny, ale powinniśmy zadbać, by w~całym wyrażeniu występował on w~tej samej postaci. Sporym ułatwieniem w~interfejsie jest brak konieczności wpisywania operatorów mnożenia ('*') i~potęgowania ('\^{}') w~oczywistych miejscach. Z punktu widzenia aplikacji, wyrażenia 4x oraz 4*x są identyczne. Podobnie jest w~przypadku x3 oraz x\^{}3. Funkcja analizuje wyrażenie i~zwraca je w~odpowiedniej postaci. Dla przykładu wyrażenia x3+2x2+3x+1, x\^{}3 +2x\^{}2 + 3x+1 zostaną zamienione w~x\^{}3+2*x\^{}2+3*x+1.

\subsubsection{Konwersja łańcucha znaków do postaci wielomianu}
\begin{lstlisting}
inline Polynomial&
 Parser::ConvertToPolynomial(string inputS, int type)
\end{lstlisting}

Funkcja ma na celu stworzenie wielomianu na podstawie wartości znakowej i~zwrócenie odpowiedniej referencji. W~zależności od parametru type, wielomian reprezentowany jest przez jeden z~dwóch rodzajów struktur. Pierwszą z~nich jest wektor -- począwszy od wyrazu stojącego przy najwyższej potędze, reprezentowane są wszystkie współczynniki, także zerowe. Drugim jest mapa -- która posiada informacje tylko o~niezerowych współczynnikach.

Na początku funkcji wykonywana jest metoda UniformInputString. Jeżeli zwróci ona pustą wartość, to funkcja przerywa swoje działanie, zwracając pusty obiekt wielomianu. W~przeciwnym wypadku następuje iteracja po kolejnych znakach łańcucha wejściowego. Ze względu na kolejność wykonywania działań, w~funkcji występują trzy obiekty wielomianów. Pierwszy z~nich posiada informacje o~aktualnie przetwarzanym fragmencie wejścia. Drugi mówi o~wartości fragmentu wielomianu, który wystąpił przed znakiem mnożenia lub dzielenia. Stanowi on, obok aktualnego wielomianu, drugi operand dla wskazanej operacji. Z kolei, trzeci jest analogiczny i~mówi o~wartości drugiego operandu dla dodawania i~odejmowania. Wszystkie te działania zostają dokonywane dopiero w~momencie, gdy natrafimy na dany operator.

W przypadku, gdy mamy do czynienia ze znakiem nawiasu otwierającego, zostaje wyszukany odpowiedni nawias zamykający, a~na wewnętrznym zakresie jest wywoływana rekurencyjnie ta sama funkcja. Zwracana wartość zapisywana jest jako aktualny element, a~wskaźnik zostaje przesunięty na kolejny. Gdy aktualnie przetwarzanym znakiem jest potęgowanie, zachowanie funkcji jest analogiczne. Wielomian zostaje podniesiony do odpowiedniej potęgi, otrzymana wartość zapisana, a~wskaźnik ustawiony na kolejny element, występujący po wykładniku.

W przypadku pozostałych operatorów działanie funkcji jest nieco odmienne. Spowodowane jest to faktem, że trafiając na dany operator, nie znamy jeszcze jego obu argumentów, a~jedynie pierwszy z~nich. Drugi jest nieznany, a~dodatkowo sam w~sobie może zawierać operatory z~wyższym priorytetem, np.\ w~przypadku wielomianu $x+3*2$ potrzebujemy najpierw obliczyć wartość wyrażenia $3*2$ i~dopiero wówczas uzyskany wynik zsumować z~pierwszym operandem, równym $x$. Fakt ten, dopiero w~momencie natrafienia na dany operator, narzuca wykonywanie poprzedniego działania. Uwaga ta tyczy się jednak tylko natrafienia na znaki dodawania, odejmowania, mnożenia i~dzielenia, ponieważ, jak zostało wspomniane wcześniej, operator potęgowania i~nawiasy posługują się osobnym algorytmem. Zachowanie to, można porównać do działania stosu, na którym mogą leżeć tylko dwa argumenty. Pierwszym z~nich jest operacja dodawania, a~drugim mnożenia. Za pomocą nich można przedstawić także odejmowanie i~dzielenie, zatem funkcje te nie będą przeze mnie rozpatrywane osobno. Operator mnożenia jest traktowany jako działanie z~wyższym priorytetem, a~dodawania z~niższym. Oznacza to, że aby wykonać sumowanie, należy zadbać o~to by ewentualna operacja mnożenia została wykonana wcześniej. Tak więc, w~momencie natrafienia na znak mnożenia lub dzielenia, wystarczy przemnożyć lub podzielić przez siebie dwa operandy. W~momencie dodawania, sytuacja jest bardziej skomplikowana. Zostaje wykonane sprawdzenie, czy odłożony został operator mnożenia. Jeżeli tak, to wartość aktualnego elementu zostaje przemnożona przez zapisany argument, w~przeciwnym razie pozostaje bez zmian. Następnie, wynik operacji zostaje dodany lub odjęty od zapisanego argumentu. W~momencie, gdy zakończymy iterowanie po całym łańcuchu wejściowym, należy zadbać o~ewentualne wykonanie mnożenia oraz dzielenia, czyniąc to analogicznie jak w~momencie rozpatrywania znaku dodawania lub odejmowania.

\subsection{Metody klasy Number}

\subsubsection{Sortowanie liczb typu mpq\_t}
\begin{lstlisting}
inline vector<Number> SortNumbers(vector<Number>v)
\end{lstlisting}

Zadaniem tej metody jest posortowanie liczb, z~biblioteki mpir, w~kolejności niemalejącej. Zarówno wejściem, jak i~wyjściem funkcji jest wektor obiektów klasy Number. Metoda ta jest używana w~celu sortowania tablicy, o~niewielkiej liczbie elementów. Oznacza to, że w~takim przypadku, bardzo dobrze sprawdzi się nieskomplikowany algorytm sortujący, np.\ sortowanie bąbelkowe. Z uwagi na mały zestaw danych, rzędu kilkudziesięciu elementów, kwadratowa złożoność czasowa, będzie akceptowalna, gdy weźmiemy pod uwagę bardzo niewielką liczbę koniecznych operacji, w~jednym obiegu pętli. Metoda skorzysta z~operatora porównania i~metody zamiany elementów klasy Number, bazujących na funkcjach mpq\_cmp oraz mpq\_swap, z~biblioteki mpir.

\subsubsection{Porównywanie wektorów liczbowych}
\begin{lstlisting}
inline int VectorsAreEqual(vector<Number>v1, vector<Number>v2)
\end{lstlisting}

Głównym celem metody jest sprawdzenie, czy dwa wektory liczbowe są sobie równe, tzn. czy zawierają wszystkie te same elementy, w~dowolnej kolejności. Funkcja przyjmuje, jako argumenty, dwa wektory, a~zwraca liczbę typu int. Jest ona równa $0$, gdy wektory są różne oraz $1$, gdy są równe. W~pierwszej kolejności porównywana jest liczba elementów obu wektorów. Gdy jest ona różna, wówczas mamy pewność, że wektory są różne. Gdy liczba elementów jest równa, elementy w~obu tablicach zostają posortowane, przy pomocy funkcji SortNumbers. Następnie porównywane są kolejne elementy wektorów. Wektory uznajemy za inne, gdy chociaż jedna para jest różna.

\subsubsection{Wyszukiwanie danej liczby w~tablicy}
\begin{lstlisting}
inline int Number::IsInVector(vector<Number> v)
\end{lstlisting}

Metoda stwierdza, czy obiekt, na którym będzie wykonywana, jest elementem wektora podanego jako argument funkcji. Zwraca ona $-1$, gdy tablica nie zawiera danej wartości liczbowej. W~przeciwnym wypadku, rezultatem funkcji jest indeks w~podanej tablicy, przy czym pierwszy jej element ma indeks równy $0$. Z uwagi, na to, że dany wektor nie zawsze jest posortowany, nie możemy skorzystać z~algorytmu wyszukiwania binarnego, posiadającego logarytmiczną złożoność, ze względu na liczbę jego elementów. By z~niego skorzystać, musielibyśmy wykonać sortowanie, a~także zapamiętywać informację o~indeksie danej liczby, w~wektorze inicjalnym. Najszybszy algorytm sortowania, wykonywany na jednym wątku, ma złożoność liniowo-logarytmiczną. Jego użycie jest wiec zupełnie nieopłacalne, jeżeli weźmiemy pod uwagę, że przejrzenie wszystkich liczb tablicy, ma dokładnie taką samą złożoność. Dodatkowo zastosowany algorytm ma własność stopu, tzn. gdy natrafi na element, którego szukamy, kończy swoje działania, zwracając odpowiedni indeks.

\subsubsection{Zwracanie wartości liczby w~postaci łańcucha znaków}
\begin{lstlisting}
inline string Number::ToString()
\end{lstlisting}

Funkcja ma celu przekształcenie danej liczby, standardowo podawanej jako ułamek, w~postaci -- licznik i~mianownik, na liczbę, zapisaną w~postaci dziesiętnej. By to osiągnąć, potrzebna jest informacja o~wymaganej precyzji obliczeń. Na początku liczba zostaje przyrównana do zera, gdy jest mniejsza, jako pierwszy znak wyniku, zostaje ustawiony minus, a~dalej liczba jest zawsze rozpatrywana jako nieujemna. Na początku zostaje obliczony iloczyn licznika i~liczby równej $10^x$, gdzie $x$ -- precyzja, wyrażona w~ważnych cyfrach po przecinku. Następnie obliczany jest całkowitoliczbowy wynik dzielenia. Do otrzymanego w~ten sposób rezultatu, wystarczy już tylko dodać separator, oddzielający część całkowitą i~ułamkową w~odpowiednim miejscu. Przypada on w~takim miejscu, by po stronie ułamkowej znajdowała się dokładnie żądana liczba cyfr. Na koniec, należy jeszcze zadbać, by otrzymany wynik był dobrze sformatowany, trzeba więc usunąć, nieznaczące zera w~części ułamkowej. Dodatkowo potrzeba pamiętać, że po wykonaniu takiej operacji, może zdarzyć się, że część ułamkowa nie zawiera już żadnych cyfr. Wówczas należy nie zapomnieć, o~usunięciu znaku separatora, tak by dla powstałej liczby całkowitej ostatnim w~kolejności znakiem była cyfra jedności.

\subsection{Metody klasy Polynomial}

\subsubsection{Największy wspólny dzielnik wielomianów}
\begin{lstlisting}
inline Polynomial&
 Polynomial::Nwd(Polynomial& p1, Polynomial& p2)
\end{lstlisting}

Jest to funkcja obliczająca największy wspólny dzielnik dwóch wielomianów. Wynik jest zwracany przy pomocy referencji do uzyskanego wielomianu. Na początku obliczana jest wartość reszty z~dzielenia ilorazu podanych wielomianów. Funkcja wołana jest w~sposób rekurencyjny. Kolejnymi argumentami funkcji dla wielomianów $p_1$ i~$p_2$ są wielomian $p_2$ oraz reszta z~dzielenia wielomianu $p_1$ przez wielomian $p_2$. Funkcja wołana jest rekurencyjnie, pod warunkiem, że otrzymana reszta, nie jest wielomianem stopnia zerowego. W~takim przypadku zwracana jest wartość $p_2$, jeżeli dzieli ona bez reszty wielomian $p_1$ lub wielomian $W(x)=1$, w~przeciwnym wypadku. Warto zauważyć, że z~każdym wywołaniem funkcji, stopień wielomianów, które są jej argumentami, maleje. Oznacza to, że dla wielomianów stopnia $n$ i~$m$, maksymalna liczba rekurencyjnych wywołań funkcji jest równa $\min(n,m)+1$.

\subsubsection{Eliminacja pierwiastków wielokrotnych wielomianu}
\begin{lstlisting}
inline Polynomial&
Polynomial::PolynomialAfterEliminationOfMultipleRoots()
\end{lstlisting}

Funkcja dla danego wielomianu, dokonuje eliminacji pierwiastków wielokrotnych i~zwraca otrzymany w~ten sposób wielomian. Na początku funkcji, obliczana jest pochodna wielomianu, a~następnie dla tych wielomianów, znajdowany jest największy wspólny dzielnik. Na otrzymywanych wielomianach dokonywana jest operacja normalizacji, tzn. podzielenie wszystkich jej współczynników, przez wartość, równą współczynnikowi, stojącemu przy najwyższej potędze. W~ten sposób otrzymamy wielomian, o~tych samych pierwiastkach, ze współczynnikiem, przy najwyższej potędze równym $1$. Pozwala to na dokonanie łatwiejszych, dzięki czemu często szybszych, obliczeń, na danych wielomianach. W~drugiej części funkcji, dokonywana jest eliminacja pierwiastków wielokrotnych, poprzez podzielenie wielomianu $W$, przez wielomian równy $NWD(W, W')$. Otrzymany wynik jest rezultatem funkcji. Zarówno dzielna, jak i~dzielnik, są wielomianami, dla których dokonaliśmy już normalizacji, wiadomo więc, że wynikowa wartość, jest już znormalizowana.

\subsubsection{Algorytm wyznaczania kolejnych przedziałów wyszukiwania}
\begin{lstlisting}
inline Number
 Polynomial::NextNumberFromRange(Number a, Number b)
\end{lstlisting}

Funkcja zwraca wybraną liczbę z~przedziału $(a,b)$. Algorytm funkcji działa tak, by zoptymalizować wybieranie przedziałów, w~których sprawdzane będzie istnienie pierwiastków. Ma to na celu jak najszybsze zawężenie przedziału, w~którym znajduje się szukany pierwiastek. Zakładamy, że długość przedziału jest niezerowa, a~liczby $a$ i~$b$ reprezentują odpowiednio -- lewy i~prawy kraniec przedziału. Z tego założenia wynika, że $a<b$ i~takich parametrów spodziewa się funkcja. W~celu optymalizacji, wewnątrz funkcji nie jest sprawdzane, czy przedstawiony warunek jest spełniony. Poniżej przedstawiona została wartość zwracana przez funkcję, w~zależności od otrzymanych na wejściu parametrów.

\[
f(a,b)=\left\{
\begin{array}{ll}
-2,\ dla\ a=-\infty \wedge b=-1\\
-1,\ dla\ a=-\infty \wedge b=0\\
0,\ dla\ a \cdot b<0\\
1,\ dla\ a=0 \wedge b=+\infty\\
2,\ dla\ a=1 \wedge b=+\infty\\
a \cdot |a|,\ dla\ a~\in (0,1) \cap (1,+\infty) \cap \{-1\} \wedge b=-\infty\ \\
b \cdot |b|,\ dla\ a=-\infty\ \wedge \ b \in (-\infty,-1) \cap (-1,0) \cap \{1\}\\
\frac{a+b}{2},\text{w pozostałych przypadkach}
\end{array}
\right.
\]

Jak widać, funkcja w~pierwszej kolejności stara się dokonać takiego podziału, by w~obu podprzedziałach znajdowały się wartości tego samego znaku. Kolejnymi wartościami, o~których należy wspomnieć są liczby $-1$ oraz $1$. Dzieje się tak z~uwagi na to, żeby zoptymalizować znajdowanie pierwiastków, zarówno takich, o~niewielkiej wartości bezwzględnej, bliskich zeru, jak i, bardzo dużych. Wówczas poprzez obliczanie kwadratu aktualnej wartości, z~zachowaniem jej dotychczasowego znaku, jesteśmy w~stanie, w~niewielkiej liczbie iteracji, maksymalnie zawęzić szukany przedział. Warto zauważyć, że dla liczb mniejszych od $1$, posiadających, jako drugi kraniec przedziału $0$, w~tempie wykładniczym zbliżamy się do $0$. Natomiast dla liczb większych od $1$, gdy drugim krańcem przedziału jest $+\infty$, w~tym samym tempie się od niego oddalamy. Przeanalizujmy jak powstają kolejne podziały w~dwóch przypadkach, gdy przedział jest równy $(-\infty,\infty)$, a~szukany pierwiastek to $-0.1$ oraz gdy przedziałem jest $(-1,30)$, a~pierwiastkiem $5$. W~celu wygody i~przejrzystości prezentowanych obliczeń, w~przykładzie przyjmijmy, że szukamy pierwiastka z~dokładnością $0.1$. Oznacza to, że w~momencie, gdy mamy przedział długości nie większej niż $0.2$, o~którym wiemy, że znajduje się w~nim pierwiastek, to jesteśmy w~stanie podać jego wartość z~żądaną precyzją, wskazując dokładnie środek danego przedziału.

\begin{example}
	$ $ \\
	$a = -\infty,\ b = +\infty\ =>\ c=0$ \\
	$a = -\infty,\ b = 0\ =>\ c=-1$ \\
	$a = -1,\ b = 0\ =>\ c=-0.5$ \\
	$a = -0.5,\ b = 0\ =>\ c=-0.25$ \\
	$a = -0.25,\ b = 0\ =>\ c=-0.0625$
\end{example}

Jak widać dla tego specyficznego przypadku, algorytm już w~5 krokach ustalił wartość poszukiwanej liczby. Długość przedziału jest równa $-0.0625-(-0.25)=0.1875<0.2$, zatem podając środek otrzymanego przedziału, równy $\frac{-0.0625+(-0.25)}{2}=\frac{-0.3125}{2}=0.15625$, mamy pewność, że będzie wskazywał on badany pierwiastek, z~wymaganą precyzją. Teraz spójrzmy na drugi scenariusz.

\begin{example}
	$a = -1,\ b = 30\ =>\ c=0$ \\
	$a = 0,\ b = 30\ =>\ c=15$ \\
	$a = 0,\ b = 15\ =>\ c=-7.5$ \\
	$a = 0,\ b = 7.5\ =>\ c=3.75$ \\
	$a = 3.75,\ b = 7.5\ =>\ c=5.625$ \\
	$a = 3.75,\ b = 5.625\ =>\ c=4,6875$ \\
	$a = 4.6875,\ b = 5.625\ =>\ c=5,15625$ \\
	$a = 4.6875,\ b = 5.15625\ =>\ c=4.921875$
\end{example}

W drugim przypadku, funkcja zakończyła działanie po 8 iteracjach. Możemy zauważyć, że liczba kroków jest zależna bezpośrednio od żądanej precyzji. Podając jako rozwiązanie kolejne środki przedziału, jesteśmy w~stanie w~czasie logarytmicznym dla badanego przedzialu, znaleźć leżący w~nim pierwiastek. Należy zwrócić uwagę, że w~niekorzystnym przypadku, kolejne przybliżenia niekoniecznie muszą być coraz bliższe szukanemu rozwiązaniu. Tak byłoby, gdy pierwiastkiem w~powyższym przykładzie była liczba 5.15. Wówczas pomimo bycia w~bliskim jego otoczeniu, kontynuowalibyśmy pracę algorytmu, a~ten znalazłby szukane rozwiązanie, dopiero kilka iteracji później. Jest to niewątpliwy minus zastosowanego algorytmu, jednak ciężki do wyeliminowania.

\subsubsection{Liczba pierwiastków wielomianu w~badanym przedziale}
\begin{lstlisting}
inline int Polynomial::NumberOfRoots(Number a, Number b)
\end{lstlisting}

Funkcja zwraca liczbę pierwiastków w~zadanym przedziale. Jest ona obliczana na podstawie liczby zmian znaku na krańcach przedziałów i~równa liczbie takich zmian na prawym z~nich, pomniejszona o~ich wartość na lewym.

\subsubsection{Znajdowanie pierwiastków wielomianu w~badanym przedziale}
\begin{lstlisting}
inline vector<Number> Polynomial::FindRoots(Number a, Number b)
\end{lstlisting}

Jest to najważniejsza funkcja całego projektu, ponieważ wewnątrz niej znajduje się cała logika projektu. Zwraca ona tablicę, z~wartościami kolejnych pierwiastków wielomianu. Na początku funkcji sprawdzana jest liczba pierwiastków w~badanym przedziale. Liczba ich nie może być ujemna, a~kiedy równa jest $0$, funkcja kończy swoje działanie i~zwraca pusty wektor.

Zgodnie z~twierdzeniem, liczba pierwiastków w~przedziale, wlicza także ewentualny pierwiastek na prawym krańcu przedziału. Gdy funkcja stwierdzi istnienie tego pierwiastka, zostaje on dodany do wektora wyjściowego. Jeżeli liczba pierwiastków w~przedziale była równa $1$, wiemy już, że był to jedyny pierwiastek w~danym przedziale i~możemy zwrócić jednoelementowy wektor, z~wartościami pierwiastków.

Następnie, na podstawie funkcji NextNumberFromRange, przedział zostaje podzielony na dwa mniejsze. W~otrzymanych przedziałach zostaje obliczona liczba pierwiastków. Jeżeli jest ona dodatnia, to następuje rekurencyjne wywołanie funkcji, z~argumentami, będącymi granicami danego podprzedziału. Wywołana rekurencyjnie metoda, po wykonaniu, zwróci wynik, który zostanie przetworzony przez funkcje wywołującą. 

\subsection{Metody czysto wirtualne klasy Polynonial – porównanie działania metod klas PolynomialMap i~PolynomialVector}

\subsubsection{Ustawianie wartości wyrazu wielomianu}
\begin{lstlisting}
inline void
    PolynomialMap::SetNumberValue(int power, Number number)
inline void
    PolynomialVector::SetNumberValue(int power, Number number)
\end{lstlisting}

Głównym zadaniem funkcji jest ustawienie wartości podanego współczynnika na daną wartość liczbową. W~przypadku mapy, konieczna jest weryfikacja, czy liczba jest równa zero. Jeżeli tak, to następuje sprawdzenie, czy w~mapie występuje już klucz, równy podanej potędzę. W~takim przypadku następuje jego usunięcie, a~gdy klucz nie istnieje, jej opuszczenie, bez jakichkolwiek dodatkowych działań. Jeżeli wartość liczbowa jest niezerowa, to sprawdzenie jest analogiczne. W~pierwszym przypadku następuje wówczas nadpisanie wartości o~danym kluczu, a~w drugim wstawienie do mapy pary – klucz (potęga), wartość (liczba).

W przypadku klasy PolynomialVector metoda została nieco bardziej rozbudowana. Na początku następuje sprawdzenie, czy dany stopień potęgi jest wyższy od aktualnego stopnia wielomianu. Jeżeli tak, to następuje sprawdzenie, czy dana wartość jest zerowa. Wówczas funkcja kończy swoje działanie, a~w przeciwnym wypadku rozpoczyna operację wstawiania do wektora żądanej pary – potęgi i~wartości współczynnika. Gdy stopień wielomianu zwiększa się o~$1$, wystarczy, po prostu, na kolejnym miejscu w~tablicy wstawić daną wartość. W~przeciwnym zaś przypadku, musimy zadbać o~to, by na miejscach wszystkich współczynników, reprezentujących ich wartości dla kolejnych potęg, począwszy od dotychczasowego stopnia wielomianu, powiększonego o~$1$, aż do jego nowego stopnia, pomniejszonego o~$1$, zostały wstawione zera. Dopiero wówczas na kolejnym miejscu w~tablicy może zostać wstawiona wartość dla odpowiedniej potęgi.

Gdy podana w~argumencie funkcji potęga, jest niewiększa od stopnia wielomianu, następuje uaktualnienie wartości na odpowiedniej pozycji w~tablicy. Konieczne jest wówczas dodatkowe sprawdzenie. Jeżeli podana para (potęga, wartość) współczynika wskazują na zerowy współczynnik dla potęgi, równej stopniowi danego wielomianu, niezbędne jest przesunięcie końca tablicy o~jedno miejsce. Następnie wykonywane są sprawdzenia kolejnych elementów funkcji, począwszy od końca. W~przypadku, gdy mamy do czynienia z~zerami, następują kolejne korekty granicy tablicy. Podana operacja trwa tak, dopóki nie zostaną przeanalizowane wszystkie elementy lub nie natrafimy na dowolną wartość różną od zera.

Warto zauważyć, że w~przypadku obu klas, funkcja ma wpływ na stopień wielomianu. Dodatkowo, operacja dokładania i~zdejmowania kolejnych argumentów, w~przypadku wektora nie została zoptymalizowana. Możliwe, że w~sytucji konieczności, wielokrotnej zmiany jej rozmiarów, korzystniejsza byłaby pojedyncza operacja, zwłaszcza w~przypadku konieczności usuwania kolejnych zerowych współczynników. Należałoby wówczas, najpierw przeanalizować kolejne ich wartości i~zliczyć wszystkie zera znaczące, a~następnie pojedynczą operacją, odpowiednio zmanipulować wskaźnik na ostatni element.

\subsubsection{Stwierdzanie czy wielomian jest wielomianem zerowym}
\begin{lstlisting}
inline bool PolynomialMap::IsZero()
inline bool PolynomialVector::IsZero()
\end{lstlisting}

Funkcja ma na celu przeanalizowanie struktury wielomianu i~stwierdzenie, czy dany wielomian jest wielomianem zerowym. Implementacja metody w~obu klasach jest analogiczna. Następuje sprawdzenie, czy wielomian posiada wyłącznie zerowe współczynniki. Zgodnie z~założeniami, w~mapie przetrzymywane są tylko niezerowe wartości wyrazów wielomianu, a~w wektorze, przechowywane są wszystkie współczynnik, aż do ostatniego niezerowego współczynnika, stojacego przy najwyższej potędze. Na tej podstawie jesteśmy w~stanie stwierdzić, że funkcja zwraca prawdę, tylko, jeżeli mapa, bądź wektor są puste.

\subsubsection{Obliczanie stopnia wielomianu}
\begin{lstlisting}
inline int PolynomialMap::PolynomialDegree()
inline int PolynomialVector::PolynomialDegree()
\end{lstlisting}

Zadaniem tej metody jest obliczenie stopnia danego wielomianu. Złożoność funkcji jest zależna od wybranej klasy. W~przypadku PolynomialMap konieczna jest analiza jej kolejnych elementów i~stwierdzenie, jaka jest największa, występująca w~niej potęga. W~przypadku PolynomialVector złożonośc funkcji jest stała, gdyż stopień wielomianu jest równy rozmiarowi tablicy, pomniejszonego o~jeden. Dla obu klas, gdy mamy do czynienia z~wielomianem zerowym, zwracaną wartością jest $-1$.

\subsubsection{Zwracanie wartości wielomianu w~postaci łańcucha znaków}
\begin{lstlisting}
inline string PolynomialMap::ToString()
inline string PolynomialVector::ToString()
\end{lstlisting}

Celem funkcji jest zwrócenie wartości wielomianu w~postaci zmiennej znakowej. Jej rezultat wygląda analogicznie, jak wynik metody UniformInputString w~klasie Parser. Na podstawie odpowiednich par – potęgi i~wartości współczynnika, zostaje stworzony obiekt typu string. Na podstawie kolejnych liczb, przed kolejnymi wyrazami, zostają dostawione odpowiednio znaki plus, dla wartości większych od $0$ oraz minus, dla wartości mniejszych. Wyjątkiem jest pierwszy wyraz, w~przypadku którego ewentualny znak minus, jest ignorowany. Pomijane są także zbędne elementy, takie jak "1*" dla współczynników, stojących przy potęgach dodatnich oraz "*x\^{}0" "\^{}1" dla odpowiednio – zerowej i~pierwszej potęgi.

Chociaż struktury różnią się reprezentacją w~pamięci, to obie zwracają identycznie sformatowany rezultat. Uznałem bowiem, że w~przypadku tablicy, przy przyjętym formatowaniu, wypisywanie zerowych współczynników jest zupełnie zbędne i~tylko uczyniłoby wynik mniej czytelnym. Opcja wypisywania tablicy z~kolejnymi współczynnikami została także odrzucona z~podobnych powodów. Sytuacja taka byłaby bardzo niekorzystna dla użytkownika. Zwłaszcza dla rzadkich wielomianów wysokich stopni, ich reprezentacja uległaby znacznej zmianie, a~ręczne znalezienie niezerowych współczyników okazałoby się praktycznie niewykonalne.

\subsubsection{Dzielenie wielomianów}
\begin{lstlisting}
inline pair<Polynomial&, Polynomial&>
    PolynomialMap::DividePolynomials(Polynomial& p1, Polynomial& p2)
inline pair<Polynomial&, Polynomial&>
    PolynomialVector::DividePolynomials(Polynomial& p1, Polynomial& p2)
\end{lstlisting}

Metoda dla podanych wielomianów oblicza ich iloraz oraz resztę z~dzielenia i~zwraca je jako parę obiektów. Na początku, funkcja oblicza wartość pierwszego wyrazu wielomianu, który zostanie zwrócony, jako iloraz. Jest on równy wynikowi dzielenia wyrazów stojących przy najwyższych potęgach wielomianów – dzielnej i~dzielnika. Kolejne wyrazy drugiego z~nich są mnożone przez otrzymaną wartość, a~wynik jest odejmowany od dzielnej. W~rezultacie tego działania, współczynnik przy najwyższej potędze staje się równy zero i~zostaje zredukowany. Daje to gwarancję, że nowy stopień wielomianu, będzie przynajmniej o~jeden mniejszy, niż stopień danego wielomianu. Funkcja kończy swoje działanie, gdy dokona powyższej operacji dla wszystkich wyrazów dzielnej. Wówczas aktualnie przetwarzany wielomian jest zwracany jako obliczona reszta. Warto zauważyć, że stopień tego wielomianu jest mniejszy od stopnia dzielnika i~co najmniej o~2 mniejszy od stopnia dzielnej.

\subsubsection{Obliczanie pochodnej wielomianu}
\begin{lstlisting}
inline Polynomial& PolynomialMap::Derivative()
inline Polynomial& PolynomialVector::Derivative()
\end{lstlisting}

Celem funkcji jest obliczenie pochodnej danego wielomianu. Jej wartość jest obliczana tylko w~przypadku, gdy wielomian jest niezerowy. Dla danego wielomianu $W(x) = a_0x^n + a_1x^{n-1} + ... + a_{n-1}x + a_n$, pochodna wielomianu wyraża wzorem: $W'(x) = na_0x^{n-1} + (n-1)a_1x^{n-2} + ... + 2a_{n-2}x + a_{n-1}$. W~przypadku klasy PolynomialMap wymagane jest przejrzenie tylko niezerowych współczynników wielomianu, zatem operacja ta jest liniowa, względem ich liczb. Z kolei, dla typu PolynomialVector konieczne jest sprawdzenie wartości wszystkich współczynników, przy czym tylko te niezerowe mają wpływ na wartość pochodnej.

\subsubsection{Obliczanie wyrazów ciagu Sturma}
\begin{lstlisting}
inline vector<PolynomialMap> PolynomialMap::GetSturm()
inline vector<PolynomialVector> PolynomialVector::GetSturm()
\end{lstlisting}

Funkcja ma na celu zwrócenie wektora wielomianów, w~którym wartościami będą kolejne wyrazy ciągu Sturma. Na początku, przy pomocy funkcji Derivative obliczana jest pochodna wielomianu. Gdy jest ona wielomianem zerowym, zostaje zwrócony ciąg Sturma, zawierający tylko jeden element – wielomian, dla którego ciąg ten jest obliczany. Wiadomo wówczas, że liczba zmian znaków dla takie ciagu, niezależnie od wybranego punktu, będzie równa zero, co oznacza, że dany wielomian nie zawiera pierwiastków rzeczywistych.

Kolejnym krokiem jest obliczenie reszty z~dzielenia wielomianu, przez jego pochodną. Jeżeli jest ona wielomianem zerowym, to zwracany jest dwuelementowy ciąg Sturma – wielomian i~jego pochodna. W~przeciwnym razie, wartość do niej przeciwna stanowi kolejny element ciągu Sturma. Podobnie jak w~przypadku pierwszej reszty z~dzielenia, także kolejne wyrazy są obliczane jako iloraz dwóch poprzednich wyrazów ciągu Sturma. Dzieje się to tak długo, dopóki otrzymany wielomian jest wielomianem zerowym. Warto zauważyć, że maksymalna liczba wyrazów ciągu Sturma jest równa stopniowi wielomianu, powiększonemu o~jeden. Wówczas stopnie kolejnych wyrazów ciągu Sturma są równe kolejnym potęgom, od stopnia wielomianu, do zera.

\subsubsection{Liczba zmian znaków w~danym punkcie}
\begin{lstlisting}
inline int PolynomialMap::NumberOfChangesSign(Number a)
inline int PolynomialVector::NumberOfChangesSign(Number a)
\end{lstlisting}

Funkcja na postawie ciągu Sturma i~metody GetSturm, oblicza liczbę zmian znaków dla podanego punktu. Porównuje ona wartości dwóch kolejnych wyrazów ciagu Sturma. Jeżeli ich iloczyn jest ujemny, następuje zwiększenie licznika o~jeden. W~przypadku wartości zerowych, element taki jest pomijany, tzn. ciąg jest rozpatrywany, tak jakby go nie zawierał. Warto zauważyć, że dla odpowiednio dużych wartości bezwzględnych liczby a, w~tym wartości niewłaściwych $-\infty$ i~$+\infty$, liczba zmian znaków, zależy wyłącznie od współczynników stojących, przy najwyższych potęgach kolejnych wielomianów. Fakt ten powoduje możliwą optymalizację działania funkcji, w~zależności od otrzymanego parametru. Najbardziej czasochłonna operacja, jaką jest dzielenie wielomianów, i~tak musi zostać wykonana, by znaleźć wszystkie wyrazy ciągu Sturma, ale zysk z~takiej optymalizacji, może być zauważalny i~znacząco poprawić wydajność funkcji.