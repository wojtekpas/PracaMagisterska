\section{Główne klasy}

\subsection{CharsConstants}

Klasa CharsConstants została zaimplementowana w~celu zapewnienia większej czytelności kodu. Ważnym aspektem było zdefiniowane i~łatwe rozpoznawanie wszystkich, mogących występować w~łańcuchu wejściowym, legalnych znaków. Takie podejście ogranicza możliwość popełnienia prostych, a~trudnych do wykrycia błędów w~przetwarzaniu otrzymanego tekstu, np.\ literówki. W~przypadku, gdy są zdefiniowane odpowiednio stałe i~funkcje, ewentualna literówka, skończy się zasygnalizowaniem błędu już na etapie kompilacji. Przypadek taki jest sygnalizowany i~łatwy do naprawienia, dzięki czemu programista nie musi zastanawiać się dlaczego, poprawny, wydałoby się, kod nie działa. Poniżej przedstawiam strukturę klasy, wraz ze zdefiniowanymi w~niej stałymi i~statycznymi metodami. 

\begin{lstlisting}
class CharsConstants
{
static const char Space = ' ';
static const char Tab = '\t';
static const char NewLine = '\n';
static const char LeastDigit = '0';
static const char GreatestDigit = '9';
static const char LeastUppercase = 'A';
static const char GreatestUppercase = 'Z';
static const char LeastLowercase = 'a';
static const char GreatestLowercase = 'z';

public:
static const char Plus = '+';
static const char Minus = '-';
static const char Mul = '*';
static const char Div = '/';
static const char Exp = '^';
static const char OpeningParenthesis = '(';
static const char ClosingParenthesis = ')';
static const char Var = 'x';

static int CharToInt(char c);
static bool IsDigit(char c);
static bool IsLetter(char c);
static bool IsUppercase(char c);
static bool IsLowercase(char c);
static bool IsWhitespace(char c);
static bool IsPlus(char c);
static bool IsMinus(char c);
static bool IsMul(char c);
static bool IsDiv(char c);
static bool IsExp(char c);
static bool IsOpeningParenthesis(char c);
static bool IsClosingParenthesis(char c);
static bool IsVar(char c);
static bool IsOperator(char c);
static bool IsLegalValue(char c);
static bool IsLegalOpeningOperator(char c);
};
\end{lstlisting}


\subsection{StringManager}

Klasa StringManager oferuje przydatne funkcje, na łańcuchach znaków. Wykorzystywana jest w~metodach klasy Parser, w~celu łatwiejszej jego implementacji oraz zwiększenia czytelności. Poniżej deklaracja klasy StringManager i~udostępnianych przez nią metod.

\begin{lstlisting}
class StringManager
{
public:
	static string EmptyString();
	static bool IsEmptyString(string s);
	static char ReturnLastChar(string s);
	static bool LastCharIsADigit(string s);
	static bool LastCharIsALetter(string s);
	static bool LastCharIsADigitOrALetter(string s);
	static bool LastCharIsAdigitOrALetterOrAParenthesis(string s);
	static int FindFirst(string s, char c);
	static int FindLast(string s, char c);
	static string Substr(string s, int first, int last);
	static vector<string> Split(string s, string operators);
	static int FindClosingParenthesis(string s);
	static string ParenthesisContent(string s);
};
\end{lstlisting}


\subsection{Parser}
Parser to klasa umożliwiająca użytkownikom podanie wielomianów poprzez standardowe wejście. Składa się tylko z~dwóch metod. Zadaniem pierwszej z~nich jest unifikacja wielomianu podanego na wejściu. Drugą metodą klasy jest tworzenie obiektu, reprezentującego wielomian. Metoda UniformInputString jest wykonywana zawsze na początku funkcji ConvertToPolynomial. Następnie na podstawie otrzymanego wyniku, funkcja ta dokonuje kolejnych obliczeń. Argument type mówi o~tym jakiego typu wielomian ma zostać stworzony wewnątrz funkcji. Zwracana referencja wskazuje na obiekt wybranego typu. Dla wartości $0$ lub $1$, wybierany jest odpowiedni typ -- PolynomialMap lub PolynomialVector. Poniżej, krótka definicja klasy Parser oraz jej metod.

\begin{lstlisting}
class Parser
{
public:
	string UniformInputString(string s);
	Polynomial& ConvertToPolynomial(string inputS, int type = 0);
};
\end{lstlisting}


\subsection{Number}
Klasa Number stanowi warstwę pośrednią, pomiędzy klasą wielomianu, a~wykorzystaniem typów liczbowych, z~biblioteki mpir. Została ona stworzona, by uniezależnić implementacje wielomianów od wykorzystanego sposobu reprezentacji dużych liczb. Dzięki temu jakakolwiek zmiana w~udostępnianych przez bibliotekę mpir, klasach i~funkcjach, wymaga zmiany kodu aplikacji, tylko w~jednym miejscu. Ma to niebagatelny wpływ na łatwość utrzymania aplikacji. Dodatkowo, wprowadzając metody opakowujące funkcje biblioteczne, możliwe jest automatyczne wykonanie dodatkowych operacji i~stworzenie nowych funkcji, ułatwiających tworzenie klas i~zwiększających czytelność kodu. Narzut czasowy związany z~koniecznością wywoływania funkcji pośrednich został oceniony jako niewielki, a~ich wpływ na przeprowadzone testy i~otrzymane rezultaty jako pomijalny. Zapoznajmy się teraz z~definicją klasy Number.

\begin{lstlisting}
class Number
{
public:
	mpq_t value;
	explicit Number();
	explicit Number(double value);
	Number(const Number &bigNumber);
	~Number();
	Number Neg();
	Number Abs();
	Number Copy();
	void SetMaxNegativeValue();
	void SetMaxValue();
	bool IsPlusInfinity();
	bool IsMinusInfinity();
	bool IsInfinity();
	bool IsVerySmallValue();
	bool IsZero();
	bool IsWithRequiredPrecision();
	int IsInVector(vector<Number> v);
	
	bool operator == (Number bigNumber);
	bool operator != (Number bigNumber);
	bool operator > (Number bigNumber);
	bool operator < (Number bigNumber);
	bool operator >= (Number bigNumber);
	bool operator <= (Number bigNumber);
	Number operator = (Number bigNumber);
	Number operator + (Number bigNumber);
	Number operator - (Number bigNumber);
	Number operator * (Number bigNumber);
	Number operator / (Number bigNumber);
	Number operator ^ (int power);
	Number operator += (Number bigNumber);
	Number operator -= (Number bigNumber);
	Number operator *= (Number bigNumber);
	Number operator /= (Number bigNumber);
	Number operator ^= (int power);
	bool operator == (double value);
	bool operator != (double value);
	bool operator > (double value);
	bool operator < (double value);
	bool operator >= (double value);
	bool operator <= (double value);
	Number operator = (double value);
	Number operator + (double value);
	Number operator - (double value);
	Number operator * (double value);
	Number operator / (double value);
	Number operator += (double value);
	Number operator -= (double value);
	Number operator *= (double value);
	Number operator /= (double value);
	string ToString();
	string MakeNice(string result);
	string RoundNine(string result);
	string TruncateZero(string result);
	void Print();
}
\end{lstlisting}

Jak widać, przeciążone zostały wszystkie przydatne dla liczb operatory, zarówno dla klasy Number jak i~dla typu double. W~języku C++ wszystkie wbudowane typy liczbowe -- int, long oraz float, można łatwo zrzutować na zmienną typu double. Wynika z~tego, że przy pomocy powyższych operatorów, jesteśmy w~stanie w~łatwy sposób, dokonać dowolnego działania na dużych liczbach, niezależnie od typu drugiego operandu. Wyjątkiem jest operator potęgowania, dla którego możliwe jest podniesienie do dowolnej potęgi, pod warunkiem, że jej wartość jest liczbą naturalną.

Klasa posiada tylko jeden element – obiekt typu mpq\_t, reprezentujący właściwą liczbę wymierną. W~klasie zostały zdefiniowane podstawowe funkcje, takie jak obliczenie wartości bezwzględnej oraz liczby przeciwnej. W~celu zwiększenia czytelności i~łatwiejszej implementacji, zostały przeciążone wszystkie przydatne operatory. Rozpatrzmy dwa przykłady użycia biblioteki mpir – pierwszy z~bezpośrednim użyciem dostępnych funkcji i~drugi z~wykorzystaniem klasy Number.

\begin{lstlisting}
mpq_t simple_function(int a, int b)
{
	mpq_t value1, value2, value3, value4;
	mpq_inits(value1, value2, value3, value4);
	mpq_set_d(value1, (double) a);
	mpq_set_d(value2, (double) b);
	
	mpq_mul(value3, value1, value2);
	if (mpq_cmp(value1, value2) > 0)
	mpq_add(value4, value1, value3);
	else
	mpq_add(value4, value2, value3);
	
	mpq_clears(value1, value2, value3);
	return value4;
}
\end{lstlisting}

\begin{lstlisting}
mpq_t simple_function_with_Number(int a, int b)
{
	Number& number1 = new Number(a);
	Number& number2 = new Number(b);
	Number& number3 = new Number(a*b)
	Number& number4 = new Number();
	
	if (number1 > number2)
	number4 = number1 + number3;
	else
	number4 = number2 + number3;
	
	delete(number1);
	delete(number2);
	delete(number3);
	return number4;
}
\end{lstlisting}

Łatwo zauważyć, że w~drugim przypadku kod jest zdecydowanie czytelniejszy. Poprzez przeciążenie operatorów, operacje na dużych liczbach wymiernych wyglądają identycznie jak działania na wbudowanych typach liczbowych. Dzięki temu, wyeliminowano bezpośrednie wywoływanie funkcji bibliotecznych. Ich użycie, pomimo charakterystycznych nazw zawsze wymagało chwili zastanowienia nad kolejno przekazywanymi argumentami. W~przypadku nieskomplikowanych funkcji, stosunkowo duży narzut związany jest z~koniecznością inicjalizowania referencji i~zwalniania miejsca, przez obiekty na które wskazują. Nie stanowi to jednak sporej zmiany, w~stosunku do przykładu pierwszego, w~którym, na początku funkcji nastąpiły deklaracja i~inicjalizacja zmiennych, a~na końcu ich zwolnienie z~pamięci.

Należy zaznaczyć, że dzięki klasie Number, nie tylko uzależniamy się od zmian wewnątrz biblioteki mpir, ale możemy także z~niej całkowicie zrezygnować. Możemy wówczas skorzystać z~alternatywnej biblioteki lub całkowicie się uniezależnić, bazując w~pełni na własnej implementacji. Ta ostatnia opcja była rozważana przeze mnie w~początkowej fazie projektu. W~wersji prototypowej powyższa koncepcja pozwoliła mi na reprezentację współczynników wielomianu, przy pomocy typu double. Jej zakres i~precyzja była wystarczająca, by aplikacja działała poprawnie dla wielomianów niskich stopni, z~ograniczoną precyzją wyszukiwania pierwiastków. Planowana była własna implementacja typu liczbowego, reprezentującego wysokie wartości dowolnej precyzji. Ostatecznie jednak koncepcja ta została odrzucona ze względów wydajnościowych. Takie rozwiązanie, napisane w~języku C++, nie mogłoby się bowiem równać z~optymalizowanym kodem biblioteki mpir, pisanym w~języku C i~posiadającym krytyczne funkcje w~postaci asemblerowych wstawek.

\subsection{Polynomial}

Klasa Polynomial to klasa abstrakcyjna, posiadająca zdefiniowane wszystkie funkcje, niezbędne do znalezienia pierwiastków danego wielomianu. Jest ona klasą bazową dla klas PolynomialMap i~PolynomialVector, reprezentujących wielomiany, przy pomocy odpowiedniej struktury. Pierwsza z~nich bazuje na mapie, czyli strukturze opartej na parach (klucz, wartość). Klucze muszą być różnowartościowe, co umożliwia jednoznaczne znalezienie wartości dla dowolnego z~nich. Pozwala to na posiadanie informacji wyłącznie o~niezerowych współczynnikach wielomianu. Druga z~nich bazuje na wektorze, jako przykładzie tablicy, której kolejne elementy są położone w~pamięci operacyjnej obok siebie. Pierwszy element tablicy to współczynnik wielomianu, stojący przy potędze zerowej, a~następne wartości to współczynniki, stojące przy kolejnych, coraz to wyższych potęgach. Dzięki temu dostęp do dowolnego wyrazu wielomianu jest bardzo szybki. Jednocześnie jednak, konieczne jest przetrzymywanie informacji o~wszystkich współczynnikach wielomianu, stojących przy kolejnych potęgach, od potęgi zerowej, aż do najwyższej potęgi z~niezerowym współczynnikiem, równej stopniowi wielomianu.

Klasa Polynomial posiada zarówno już napisane metody, jak i~takie, które zostały tylko zaprojektowane i~oznaczone jako czysto wirtualne, czyli konieczne do zaimplementowania w~klasie dziedziczącej. Do tego pierwszego zbioru zalicza się część metod, która nie odnosi się do konkretnej struktury wielomianu. Dzięki temu, w~klasach podrzędnych nie ma konieczności ponownego ich pisania. Posiadanie jednej implementacji zamiast dwóch pozwalało na łatwiejszą i~dużo sprawniejszą konstrukcję obu klas dziedziczących. Spore zmiany projektowe i~niewielkie korekty wystarczyło wprowadzić tylko raz, zamiast niepotrzebnie je powielać. Niestety, w~przypadku sporej części funkcji, mimo bardzo zbliżonej implementacji, nie było możliwe ich połączenie i~umieszczenie w~klasie bazowej. Spowodowane to było koniecznością odwołań do konkretnego typu danych, w~którym zostały umieszczone wyrazy wielomianu. Uogólnione są natomiast wszystkie pozostałe funkcje, w~tym także te, które posiadają wywołania funkcji czysto wirtualnych. W~przypadku tym, zastosowany został polimorfizm, czyli jeden z~paradygmatów programowania obiektowego. Pozwala on odwoływać się do zdefiniowanej metody, bez znajomości jej implementacji. Wywołana funkcja zadziała różnie, zależnie od typu obiektu, na którym zostanie wykonana. Z punktu widzenia poprawnego działania klasy Polynomial, bez znaczenia jest implementacja operatora przypisania, o~ile spełnia on swoją rolę. W~ten sposób możliwe jest działanie na obiekcie wielomianu, niezależnie, czy bazuje on na tablicy, czy mapie.

W języku C++, by można zastosować polimorfizm, poszczególne metody muszą operować na wskaźnikach lub referencjach. Nie możliwe jest użycie tego mechanizmu w~przypadku przekazywaniu obiektów poprzez wartość. Wytłumaczenie tego jest bardzo proste i~opiera się na zmiennym rozmiarze obiektów klas pochodnych. Uniemożliwia to odwołanie się do danego obiektu, póki nie znamy jego dokładnego typu. Inaczej jest w~przypadku referencji i~wskaźników, ponieważ ich rozmiar jest stały, a~zmieniać się może jedynie ich wartość, czyli miejsce w~pamięci, do którego się odwołują.

Z uwagi na powyższy fakt, w~programie wszystkie obiekty klasy Polynomial są przekazywane poprzez referencję. By referencję, można przekazywać pomiędzy funkcjami, obiekt na który wskazuje, musi zostać stworzony dynamicznie. Można to zrobić poprzez użycie, znanej z~języka C, funkcji malloc, alokującej miejsce w~pamięci lub typowego dla języka C++, operatora new. W~przeciwnym wypadku, zmienna będzie widoczna tylko w~miejscu, w~którym zostanie stworzona, np.\ wewnątrz funkcji. Próba odwołania się do takiej wartości, w~miejscu, w~którym zmienna nie jest widoczna, skończy się błędem czasu wykonania i~komunikatem o~naruszeniu dostępu. Stanie się tak, ponieważ referencja w~takim przypadku będzie dalej istnieć, ale obiekt, na który wskazuje, już nie. Takie błędy często są popełniane przez niedoświadczonych programistów, a~próba ich lokalizacji i~znalezienia przyczyny, na pierwszy rzut oka, nie jest oczywista. W~momencie, gdy skorzystamy z~dynamicznego tworzenia obiektu, będzie on istniał tak długo, dopóki nie zostanie jawnie usunięty w~kodzie programu lub aplikacja nie skończy swojego działania, zwalniając przy tym całą zajmowaną pamięć. Kiedy nie wszystkie stworzone obiekty zostają usunięte, dochodzi do tzw. wycieków pamięci. W~zależności od ich rozmiarów i~czasu działania aplikacji, ich konsekwencje mogą być bardzo różne. W~skrajnym wypadku, może dojść do wyczerpania całej dostępnej pamięci. Gdy zaczyna jej brakować, system operacyjny zapisuje jej ostatnio nieużywaną część na dysku, by w~razie potrzeby móc ją odczytać. Z uwagi na to, że pamięć operacyjna jest wielokrotnie szybsza od dysków twardych, taka operacja powoduje gigantyczne opóźnienia w~pracy komputera. Gdy szybkość kolejnych alokacji pamięci jest większa, niż jej zrzucanie na dysk, w~pewnym momencie komputer ulegnie całkowitemu zawieszeniu. Wówczas jedynym wyjściem jest, często bardzo niepożądany, restart systemu. Bezpiecznym rozwiązaniem jest ustalenie limitu na wykorzystanie przez pojedynczą aplikację. Może to znacznie ułatwić debugowanie i~znalezienie ewentualnego błędu, a~także zabezpieczy użytkownika przed uruchomieniem złośliwego oprogramowania, mającego na celu doprowadzenie do wspomnianej wyżej sytuacji.

Poniżej przedstawiam definicję klasy abstrakcyjnej Polynomial. Wyróżnić w~niej można kolejne sekcje -- zmienne klasy, konstruktory, funkcje i~operatory. Te ostatnie możemy podzielić pomiędzy te, które posiadają implementacje w~klasie abstrakcyjnej Polynomial oraz te, które zostały zdefiniowane jako czysto wirtualne i~konieczne jest ich nadpisanie w~klasie pochodnej.

\begin{lstlisting}
class Polynomial
{
public:
	MAP m;
	VECTOR v;
	bool isNew = true;
	int type = 0;
	string inputS = "";
	vector<Number> roots;
	int id = 0;
	
	explicit Polynomial();
	explicit Polynomial(Number number);
	~Polynomial();
	
	virtual Polynomial& CreatePolynomial() = 0;
	virtual Polynomial& CreatePolynomial(Number number) = 0;
	virtual void Clear() = 0;
	virtual bool IsZero() = 0;
	virtual int Size() = 0;
	virtual int PolynomialDegree() = 0;
	virtual Number Value(int power) = 0;
	virtual pair<Polynomial&, Polynomial&> \
	DividePolynomials(Polynomial& p1, Polynomial& p2) = 0;
	virtual void SetNumberValue(int power, Number number) = 0;
	virtual int NumberOfChangesSign(Number a) = 0;
	virtual Polynomial& NegativePolynomial() = 0;
	virtual Polynomial& Derivative() = 0;
	virtual Number PolynomialValue(Number a) = 0;
	virtual string ToString() = 0;
	virtual void SturmClear() = 0;
	virtual int TheLowestNonZeroValue() = 0;
	
	virtual bool operator==(Polynomial& p2) = 0;
	virtual Polynomial& operator = (Polynomial& p2) = 0;
	virtual Polynomial& operator + (Polynomial& p2) = 0;
	virtual Polynomial& operator - (Polynomial& p2) = 0;
	virtual Polynomial& operator * (Polynomial& p2) = 0;
	
	virtual VECTOR VectorValuesExceptValueOfPolynomialDegree
	    (int degree) { return{}; };
	virtual MAP MapValuesExceptValueOfPolynomialDegree
	    (int degree) { return{}; };
	
	bool Set(string s);
	bool IsNew();
	PAIR ValueOfPolynomialDegree();
	
	bool ValueEquals(int power, Polynomial& p2);
	void SetValue(int power, double value);
	void Add(int power, Number number);
	void Sub(int power, Number number);
	PAIR Mul(int power1, Number number1, 
	      int power2, Number number2);
	PAIR Div(int power1, Number number1,
	      int power2, Number number2);
	Polynomial& Nwd(Polynomial& p1, Polynomial& p2);
	Polynomial& PolynomialAfterEliminationOfMultipleRoots();
	void Normalize();
	Number CoefficientValue(PAIR pair1, Number a);
	Number NextNumberFromRange(Number a, Number b);
	int NumberOfRoots(Number a, Number b);
	int AddNextRoot(Number x);
	vector<Number> FindRoots(Number a, Number b);
	void PrintRoots(Number a, Number b);
	
	bool operator!=(Polynomial& p2);
	Polynomial& operator / (Polynomial& p2);
	Polynomial& operator % (Polynomial& p2);
	Polynomial& operator ^ (int power);
	Polynomial& operator += (Polynomial& p2);
	Polynomial& operator -= (Polynomial& p2);
	Polynomial& operator *= (Polynomial& p2);
	Polynomial& operator /= (Polynomial& p2);
	Polynomial& operator %= (Polynomial& p2);
	Polynomial& operator ^= (int power);
	void Print();
	void PrintInput();
};
\end{lstlisting}

\subsection{PolynomialVector}

Klasy PolynomialMap i~PolynomialVector są implementacjami klasy abstrakcyjnej Polynomial. Obie one nadpisują wszystkie czysto wirtualne metody klasy bazowej. Wiele z~tych metod wygląda bardzo podobnie, natomiast istotną różnicą jest konieczność odwołania się do konkretnej reprezentacji danych. To właśnie dlatego zostały one oznaczone w~klasie bazowej, jako metody abstrakcyjne. Takie zaprojektowanie klasy, wymusza na użytkowniku korzystanie z~interfejsu, udostępnionego przez klase bazowe. Jego poprawne zaimplementowanie daje gwarancje, że klasa będzie wykonywać to czego oczekuje bazowa, nie narzucajac sposóbu, w~jaki ma to robić.

Jak zostało wspomniane, różnicą obu klas jest podejście do współczynników zerowych. Koncepcja reprezentacji wszystkich z~nich została oparta na typie vector, z~biblioteki STL. Tablica zakłada, że kolejne współczynniki bedą zapisywane w~jej kolejnych komórkach, przy czym indeks w~tablicy, począwszy od zerowego, określa potęgę dla współczynnika, którego wartość jest tam zapisana. Aplikacja musi pamiętać wskaźniki na pierwszy i~ostatni element. Na ich podstawie jest ona w~stanie stwierdzić, ile elementów zawiera, co w~sposób bezpośredni przekłada się na stopień danego wielomianu. Największym mankamentem takiego podejścia jest konieczność posiadania informacji, o~wszystkich współczynnikach, także zerowych. W~przypadku wielomianów rzadkich, o~wysokim stopniu, np.\ $x^{1000}-1$, konieczne jest przechowywanie 999 zer w~kolejnych komórkach. Poza negatywnym wpływem na złożoność pamięciową takiego podejścia, istotniejszym wydaje się fakt, że znalezienie wyłącznie niezerowych elementów, wymaga przejrzenia wszystkich komórek tablicy. Zatem, by dodać do siebie dwa wielomiany $x^{1000}$ oraz $2x^{1000}$, konieczne jest wykonanie aż $1001$ sumowań lub sprawdzenie dla każdego z~nich, czy jest potrzebne.

Dodatkowo, w~przypadku, gdy w~wyniku działania, zmienia się stopień wielomianu, konieczne jest rozszerzenie lub pomniejszenie tablicy, poprzez aktualizacje odpowiednich wskaźników. Jeżeli chodzi o~złożoność czasową funkcji, nie ma to jednak wpływu, ponieważ dla wszystkich typów operacji, i~tak jesteśmy zmuszeni przejrzeć wszystkie elementy. Wyjątkiem jest stwierdzenie, czy mamy do czynienia z~wielomianem zerowym, którego złożoność jest stała, gdyż wartość ustalana jest na podstawie rozmiaru tablicy.

Jeżeli chodzi o~zalety korzystania z~tablicy, jako struktury do przechowywania współczynników wielomianu, to najważniejszy jest stały czas dostępu do dowolnego z~nich, poprzez możliwość odwołania się do konkretnego elementu tablicy. W~przypadku zapisu pod wskazany adres, czas ten może się wydłużyć, w~przypadku zmiany stopnia wielomianu. Widać więc, że użycie tablicy wydaje się uzasadnione w~przypadku operacji na wielomianach gęstych, których stopień wielomianu nie zmienia się zbyt często. Struktura ta doskonale sprawdzi się do zsumowania dwóch wielomianów, stopnia setnego, zawierających wszystkie współczynniki równe $1$. Z kolei, dla przypadku, gdy wielokrotnie sumujemy rzadkie wielomiany, o~przeciwnych współczynnikach, wydajność tej struktury prawdopodobnie nie będzie zadowalająca.

\subsection{PolynomialMap}

Klasa PolynomialMap zakłada strukturę wielomianu, reprezentującą wyłącznie niezerowe współczynniki.
Pozwala to zaoszczędzić miejsce w~pamięci, dla przechowywania wartości zerowych i~łatwo stwierdzić ile ich jest. Również w~tej klasie wykorzystałem typ z~biblioteki STL, jakim jest map. Mój wybór był spowodowany faktem, że jest to rodzaj kontenera danych, w~którym mamy łatwy dostęp do żądanej wartości. Nie jest on tak szybki, jak wektor, bo w~każdym zapytaniu, musimy dowiedzieć się o~położeniu konkretnej wartości. Mapa nie gwarantuje nam, że trzymane w~niej dane będą znajdować się, w~położonej blisko siebie pamięci. Dzięki temu, modyfikacja jej rozmiaru, jest bardzo łatwa, gdyż wystarczy zmienić tylko te wartości, które rzeczywiście się zmieniają. Gdy wielomian $x^{100} + 1$, zamieniamy na $x^{10} + 1$, wystarczy, jedyne co musimy zrobić, to usunąć wartość dla klucza równego $100$, a~dodać klucz równy $10$, z~wartością $1$.

W porównaniu z~PolynomialVector, klasa ta wydaje się lepsza dla rzadkich struktur wielomianów, a~gorsza dla gęstych. Przeanalizujmy krótko, dlaczego tak podpowiada intuicja. Niepodważalne jest, że pojedynczy dostęp do dowolnego elementu, w~przypadku tablicy jest szybszy, niż dla jakiejkolwiek innej struktury. Dzieje się tak, ponieważ na podstawie miejsca początku tablicy, dla dowolnego jej elementu, na podstawie indeksu, z~góry znamy jego dokładny adres. W~przypadku wszystkich kontenerów danych, których kolejne elementy, nie są położone w~pamięci obok siebie, najpierw musimy je znaleźć, tj.\ uzyskać ich adres.

Rozważmy strukturę danych, jaką jest drzewo. Jego elementy są posortowane, ale bezpośredni dostęp mamy tylko do korzenia. Do innych elementów, można dostać się w~sposób, wyłącznie pośredni. Niezależnie od rodzaju drzewa, jaki wybierzemy, nie jesteśmy w~stanie przekroczyć pewnych charakterystycznych dla niego wartości. W~przypadku drzewa, średnia złożoność, znalezienia wybranej wartości jest logarytmiczna. Mowa tu, o~najkorzystniejszym wariancie, czyli przypadku, gdy drzewo to jest zrównoważone, tzn. takie, w~którym, odległość do jego liści jest różniąca się maksymalnie o~jeden i~bliska wartości logarytmu z~rozmiaru drzewa. Dodatkowo każde dodanie, modyfikacja i~usunięcie dowolnego elementu, zaczyna się od jego znalezienia. Oznacza to, że złożoność logarytmiczna jest w~takim przypadku, wartością graniczną.

Wydaje się więc, że aby uzyskać w~przypadku takiej struktury, wydajność lepszą niż dla tablicy, liczba ich elementów musi się odpowiednio różnić. Potrzebuje ona mieć odpowiedni zapas liczby operacji, by ich czas, pomimo długości każdej z~nich, był łącznie mniejszy.	
