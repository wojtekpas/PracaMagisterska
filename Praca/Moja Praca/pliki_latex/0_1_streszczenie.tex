\chapter*{Streszczenie}
\indent
Niniejsza praca ma na celu przedstawienie i omówienie sposobu pozwalającego na znajdowanie pierwiastków rzeczywistych wielomianu w określonym przedziale. Najważniejszym aspektem był dobór odpowiedniej metody tak, by była skuteczna w każdym przypadku, a~nie tylko w określonych warunkach. Przykładem takiego sposobu jest wykorzystanie twierdzenia Sturma. Przed jego zastosowaniem należy dokonać eliminacji pierwiastków wielokrotnych wielomianu, by móc następnie zdefiniować ciąg Sturma. Na jego podstawie jesteśmy w stanie określić liczbę różnych pierwiastków w dowolnym przedziale.  Na mocy twierdzenia jest ona równa różnicy liczby zmian znaku na obu jego krańcach. Badany przedział możemy więc dowolnie zmniejszać, by z wymaganą precyzją wskazać wartość szukanych pierwiastków.

Ważnym aspektem pracy jest aplikacja konsolowa w języku C++, pozwalająca na znajdowanie pierwiastków wprowadzonego przez użytkownika wielomianu. Do jej wykonania niezbędna jest implementacja działań na wielomianach oraz obsługa dużych wartości liczbowych o wysokiej precyzji. W tym celu program korzysta z biblioteki MPIR. By zapewnić odpowiednią jakość, powstał osobny moduł aplikacji, w którym zostały zdefiniowane odpowiednie testy funkcjonalne programu.

Ciekawym zagadnieniem pracy jest porównanie różnych typów struktur, pozwalających na reprezentację wielomianu. Pierwszą z nich jest tablica, w której trzymane są wszystkie współczynniki wielomianu. Drugą zaś jest mapa, jako przykład struktury, o braku bezpośredniego dostępu do wszystkich jej elementów, co pozwala na posiadanie informacji wyłącznie o niezerowych współczynnikach. Przeprowadzone zostały testy porównujące wydajność działania obu struktur. Wyniki obu z nich zostały porównane zarówno z wartościami teoretycznymi, jak i ze sobą. Dodatkowo została przeprowadzona analiza, która ze strukturach jest lepsza dla różnych typów wielomianów.

\vspace{12pt}
\noindent\textbf{Słowa kluczowe:}

wielomian, pierwiastki wielomianu, twierdzenie Sturma, biblioteka mpir

\vspace{12pt}
\noindent\textbf{Dziedzina nauki i techniki, zgodnie z wymogami OECD:}

Nauki przyrodnicze, Matematyka, Nauki o komputerach i informatyka
