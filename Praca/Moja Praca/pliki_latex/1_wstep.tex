\chapter{Wstęp}

Celem niniejszej pracy jest implementacja algorytmu, pozwalającego na obliczanie pierwiastków rzeczywistych dowolnego wielomianu. Jednym z~najlepszych sposobów na zrobienie tego jest skorzystanie z~twierdzenia Sturma. Planuję zbadać, jaki sposób reprezentacji wielomianu jest najbardziej wydajny. Dodatkowo zamierzam także porównać możliwości stworzonej przeze mnie aplikacji z~innymi programami.

Zacznę od przedstawienia tego, czym są wielomiany. Zapoznanie się bowiem z~podstawowymi definicjami i~twierdzeniami jest moim zdaniem niezbędne, by dobrze zrozumieć badany problem i~móc wypracować jego skuteczne rozwiązanie. Warte dość szczegółowego opisu są działania charakterystyczne dla wielomianów. Są one analogiczne jak w~przypadku liczb. Należą do nich dodawanie, odejmowanie, mnożenie, dzielenie, obliczanie reszty z~dzielenia. Dwa ostatnie są operacjami najbardziej skomplikowanymi i~tym samym najbardziej złożonymi pod względem czasowym. Dodatkowo warto zaznaczyć, że w~przypadku wielomianu możliwe jest obliczanie pochodnej, największego wspólnego dzielnika oraz najmniejszej wspólnej wielokrotności.

Jak wspomniałem na początku, w~pracy wykorzystam metodę ciągu Sturma. Pozwala ona znaleźć wszystkie pierwiastki rzeczywiste danego wielomianu, występujące w~podanym przedziale. Zakłada ona zbudowanie ciągu wielomianów, coraz niższych stopni, a~następnie porównywanie ze sobą ich wartości w~danym punkcie. Na tej podstawie zostaje obliczona liczba zmian znaku w~ciągu Sturma. Różnica tych wartości w~obu krańcach badanego przedziału jest liczbą pierwiastków rzeczywistych, które w~nim występują. W~przeciwieństwie do wielu innym metod, twierdzenie Sturma jest skuteczne dla każdego wielomianu. Jedynym ograniczeniem jest konieczność dokonania eliminacji pierwiastków wielokrotnych danego wielomianu. By to uczynić, wystarczy podzielić go przez największy wspólny dzielnik tego wielomianu oraz jego pochodną.

W przypadku obliczeń na wielomianach ważnym aspektem są obliczenia na bardzo dużych liczbach z~dowolną precyzją. Konieczne jest to zwłaszcza, gdy mamy do czynienia z~wielomianami wysokich stopni. Wówczas każda liczba, której wartość bezwzględna jest większa od jedynki, po podniesieniu do wysokiej potęgi, jest bardzo duża. Dla przykładu wielomian $x^{100}$ w~punkcie $x_1=10$ ma wartość $10^{100}$. Z kolei w~przypadku pozostałych liczb, wartość ta bardzo szybka zbliża się do zera – dla $x_2=0.1$ wartość dla tego samego wielomianu będzie wynosić zaledwie $(0.1)^{100}$. Aplikacja jest implementowana w~języku C++, a~żaden z~wbudowanych w~niego typów nie posiada możliwości przechowywania, aż tak dokładnych i~na tyle dużych wartości. Rozwiązaniem jest tutaj skorzystanie z~biblioteki dużych liczb, z~dowolną precyzji – mpir. Posiada ona swoją implementację dla wielu języków programowania, w~tym C i~C++.  Warto zaznaczyć, że rozwiązanie to jest bardzo wydajne, ponieważ najbardziej krytyczne funkcje zostały zaimplementowane w~asemblerze, a~cała biblioteka jest optymalizowana pod kątem użycia na konkretnym procesorze.
Projekt zakłada istnienie trzech głównych modułów. Pierwszym z~nich jest biblioteka statyczna, w~której zawarte będą wszystkie niezbędne klasy i~funkcje. Drugim jest aplikacja konsolowa, która pozwoli na interakcję użytkownika programu z~biblioteką i~wykonywaniem jej metod dla konkretnych wielomianów. Zaś ostatnim są testy funkcjonalne, pozwalające na przetestowanie funkcjonalności programu, w~celu zapewniania mu odpowiedniej jakości i~łatwego monitorowania jego poprawności.
Przewiduję implementację wielomianów, bazującą na dwóch odmiennych strukturach. Pierwszą z~nich jest tablica, która wymusza reprezentację wszystkich jego współczynników, także zerowych. Druga koncepcja zakłada przechowywanie informacji wyłącznie o~niezerowych wartościach  współczynników wielomianów. Opierać się ona będzie na drzewiastej strukturze, w~której brak jest bezpośredniego dostępu do dowolnego elementu. Do reprezentacji powyższych struktur planuję wykorzystać typy z~biblioteki SDL, jakimi są wektor oraz mapa.

W dalszej części pracy zawarty będzie opis wszystkich stworzonych funkcji i~klas, w~tym dokładna analiza porównawcza obu typów wielomianów. Planowane jest stworzenie abstrakcyjnej klasy, posiadającej metody, które będą całkowicie niezależne od reprezentującej dany wielomian struktury. Wszystkie pozostałe metody będą w~niej zadeklarowane i~oznaczone jako czysto wirtualne. Zgodnie z~paradygmatami programowania obiektowego wymagane będzie przeciążenie każdej z~nich by móc stworzyć instancję klasy danego wielomianu. Poprzez zdefiniowanie odpowiedniego interfejsu, łatwiejsza będzie także potencjalna rozbudowana aplikacji o~ewentualne zdefiniowanie kolejnych typów wielomianów, bazujących na innych niż wymienione wyżej struktury.

Do programu zostanie dołączona dokładna instrukcja programu oraz niezbędne informację o~wymaganym środowisku. Warto zaznaczyć, że celem projektu jest stworzenie aplikacji działającej pod systemem Windows. Nie jest natomiast planowane jej uruchomienie pod Linuksem. Należy także zaznaczyć, że wykorzystywana biblioteka mpir będzie skompilowana dla procesorów Intel i7 i~wykorzystanie aplikacji używając innego sprzętu może być niemożliwe lub wymagać wykonania dodatkowych czynności.
Jak zostało wspomniane wcześniej, projekt będzie posiadał moduł pozwalający na zdefiniowanie i~wykonywanie testów funkcjonalnych. Rozwiązaniem, które to umożliwia, jest framework testów jednostkowych, wbudowany w~narzędzie deweloperskie Microsoft Visual Studio 2015. Napisane testy będą dotyczyć zarówno krytycznych funkcji programu, jak i~weryfikować poprawne działanie całego projektu.

Zostaną przeprowadzone testy wydajnościowe, które porównają działanie czasowe obu typów wielomianów. Rozpatrzone będą specyficzne przypadki testowe, faworyzujące poszczególne reprezentacje oraz testy dla losowo dobranych współczynników wielomianów. Nastąpi analiza, na podstawie której, zostaną wyciągnięte wnioski, w~jakich przypadkach każda z~testowanych struktur jest bardziej wydajna. Otrzymane wyniki zostaną porównane z~wartościami teoretycznymi dla poszczególnych typów wielomianów. Dodatkowo możliwości programu zostaną porównane z~inną aplikacją, pozwalającą znajdować pierwiastki rzeczywiste wielomianów.
