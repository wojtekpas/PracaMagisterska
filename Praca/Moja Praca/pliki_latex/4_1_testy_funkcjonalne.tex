\chapter{Przeprowadzone testy}

\section{Testy funkcjonalne}

Przeprowadzone testy są połączeniem testów jednostkowych i funkcjonalnych. Zastosowałem nieco odmiennie podejście od klasycznego, tzn. w moim przypadku najpierw powstał rdzeń aplikacji i dopiero wówczas zaczęły się pojawiać pierwsze testy. Spowodowane było to faktem, że dopiero wraz z rozwojem projektu zdałem sobie sprawę, że są one niezbędne do zapewnienia odpowiedniej jakości aplikacji, poprzez stałą weryfikację jej działania. Dodatkowo, regularna egzekucja istniejących już testów daje twórcy aplikacji informację o ewentualnej regresji i możliwość szybkiej korekty wprowadzonych zmian.

Testy funkcjonalne często zwane są testami czarnej skrzynki i wykonywane nie przez twórców aplikacji, a przez zewnętrznych testerów. W tym przypadku jest jednak inaczej i służą one wyłącznie programiście. Do tego celu nie zostało wykorzystane żadne zaawansowane środowisko testowe. Użyłem do tego framework, wbudowany w Microsoft Visual Studio 2015, służący do pisania testów jednostkowych. Okazał on się bardzo intuicyjny i prosty w obsłudze, a do mojego zastosowania całkowicie wystarczający. Do wszystkich testów wykorzystałem możliwość łatwego porównywania wartości typów string. Pozwoliło mi to na łatwe debugowanie testów i wyraźna informację, o tym jaka wartość była oczekiwana, a jaka została zwrócona przez wołaną funkcję. Ważnym aspektem było to, że wszystkie testowane klasy i funkcje aplikacji, w łatwy sposób, mogą być rzutowane na typ znakowy.

Testowane funkcje i powstałe klasy testowe zostały wybrane tak, by w jednoznaczny sposób móc zlokalizować przyczynę błędu. Rzeczywistość szybko zweryfikowała to założenie, nie mniej problemem była bardzo nieoczywista przyczyna, a nie źle dobrane testy. W pozostałych przypadkach wyniki testów zawsze powinny być czytane w ściśle określonej kolejności. Została ona ustalona tak, by funkcje niezależne od innych były wykonywane najpierw, a dopiero w przypadku ich poprawnego działania, następowała analiza tych bardziej złożonych. Tak więc, na początku testowane jest parsowanie łańcucha znaków do postaci wielomianów, następnie poszczególne operatory działań na wielomianach, by na końcu przeprowadzić weryfikację zasadniczej części programu, tzn. znajdowania pierwiastków.

Przeprowadzane testy programu weryfikują poprawne działanie obu struktur jednocześnie. Innymi słowy test daje wynik pozytywny wyłącznie wtedy, kiedy zwrócony wynik w przypadku obu typów PolynomialMap i PolynomialVector jest poprawny. Jest to parametr stosunkowo łatwy do zmienienia w kodzie programu, ale ustaliłem, że taka postać testów będzie najbardziej optymalna. W trakcie dewelopmentu aplikacji i debugowanie testy zostały zmieniane tak, by testować osobną daną funkcjonalność dla konkretnego typu wielomianu. Był to niezbędny krok, by komunikat zwracany przez framework testowy był jasny i czytelny. Dodatkowo warto zauważyć, że informacja o tym, że tylko jedna ze struktur działa, dostarcza wiele danych programiście. Dzięki temu, lista przyczyn usterki jest mocno ograniczona, co zazwyczaj bardzo przyśpiesza jej znalezienie i naprawienie.

\subsection{ParserUniform}

Klasa testowa pozwalająca na weryfikację metody UniformInputString, mającej na celu unifikację otrzymanej wartości znakowej, przedstawiającej wielomian. Testy dla niej zostały napisane, pomimo że funkcja nie jest skomplikowana. Jej funkcjonalność została uznana za podstawową, a jej poprawne działanie jest niezbędne, by korzystać z programu. Jest to jedyna funkcja, która jest wołana niezależnie od innych.

Jej zadaniem jest nie tylko ujednolicić wprowadzaną postać wielomianu, ale przede wszystkim dokonać walidacji jej poprawności. Musi być ona odporna na różnego rodzaju błędy, wprowadzane przez użytkownika, zarówno przypadkiem, jak i celowo. Jej celem jest przeanalizować i zwrócić pusty obiekt typu string, w momencie, gdy nie uda się jej zrozumieć i w pełni przeanalizować otrzymanego wejścia. Funkcja ConvertToPolynomial wywołuje ją na początku, a następnie bazując na otrzymanym rezultacie, tworzy wybrany obiekt typu PolynomialMap lub PolynomialVector.

Testy tej funkcji zakładają sprawdzenie, że dla danego wejścia, zostaje zwrócone poprawne wyjście, czyli takie, które jest standardowe i łatwe do sparsowania. W przeciwnym wypadku, zwrócony ma zostać pusty łańcuch. Niezależnie jakie dane zostaną podane na wejściu, program nie ma prawa zakończyć się błędem, rzucając wyjątek, ani w ogóle się nie skończyć.

Na przedstawioną funkcjonalność napisałem ponad $20$ testów funkcjonalnych. Poniżej zamieszczam przykładowe z nich, przedstawiając je w postaci -- wejście, wyjście. 

\begin{lstlisting}
"" -> ""
"2x*+4" -> ""
"3*x-2" -> "3*x-2"
"4x" -> "4*x"
"x2+5" -> "x^2+5"
"10x3 - 4 x2 + 5x" -> "10*x^3-4*x^2+5*x"
"(3x-1)(2x+4)+2(x-4)" -> "(3x-1)*(2x+4)+2*(x-4)"
\end{lstlisting}

\subsection{ParserConvert}

Jest to klasa pozwalająca na testowanie metody ConvertToPolynomial. Wywołuje ona funkcję UniformInputString i analizuje jej wartość. Gdy jest to pusta wartość, tworzy wielomian zerowy. W przeciwnym wypadku, analizuje ona uzyskany rezultat i na jego podstawie tworzy wybrany obiekt typu PolynomialMap lub PolynomialVector. Rodzaj stworzonego wielomianu jest podawany jako parametr funkcji. W zależności od niego, także wszystkie pośrednie wielomiany, przybierają żądaną formę.

Przetestowanie tej funkcji powinno być możliwie bardzo dokładne i zawierać różne przypadki graniczne. Bez poprawnego działania tej funkcji, użytkownik w żaden sposób nie może przewidzieć jaką wartośc będzie miał dany wielomian. Ewentualny błąd w tej funkcji może powodować nieprzewidziane zachowanie pozostałych. W zachowaniu funkcji uwzględnione przypadki niedopuszczalnych wielomianów, które jednak przeszły weryfikację poprawności funkcji UniformInputData. Są nimi między innymi: ujemna lub niecałkowita potęga wielomianu, czy dzielenie przez wielomian zerowy. Wejścia takie nie zostaną odrzucone, ponieważ zawierają one wielomiany poprawne składniowo, a nielegalne wartości zostaną wykryte dopiero w momencie dokonania obliczeń, wykonanywanych jako pierwsze, np.\ z powodu nawiasu.

By przetestować funkcję, napisałem blisko $50$ różnych testów. Poniżej przedstawiam wybrane z nich przypadki testowe.

\begin{lstlisting}
"123" -> "123"
"100a" -> "100a"
"a^2" -> "a2"
"-5*a3" -> "-5a3"
"4*a+2" -> "4a+2"
"(3*a)^2" -> "9a2"
"(-2)10" -> "1024"
"a*2a" -> "2a2"
"(a+1)(a-1)" -> "a2-1"
"(a+1)(a+1)" -> "a2+2a+1"
"(a-1)^3" -> "a3-3a2+3a-1"
"(a3-1)*0" -> "0"
"(a-1)^3+4a2-a2+0*(a3-1)" -> "a3+3a-1"
"5^0" -> "1"
"(2a-3a+a)^1" -> "0"
\end{lstlisting}

\subsection{Operatory dodawania, odejmowania i mnożenia}

\subsubsection{PolynomialSumOperator}

Jest to klasa testowa odpowiedzialna za testowanie funkcjonalne operatora dodawania. W klasie Polynomial został przeciążony operator sumy, pozwalający dodawać dwa wielomiany, zapisując działanie to w czytelny i przejrzysty sposób. Testowanie poprawności tego działania polega na wywołaniu funkcji weryfikującej z trzema parametrami. Dwa pierwsze są składnikami dodawania, a ostatni to oczekiwana suma. Funkcja za pomocą operatora dodawania oblicza sumę wielomianów, a następnie porównuje ją z oczekiwanym wynikiem. Działanie polega na dodaniu wartości wyrazów, o tych samych potęgach, czyli redukcji wyrazów podobnych. Poniżej zamieszczam przykładowe testy.

\begin{lstlisting}
"1" + "20" = "21"
"5" + "-11" = "-6"
"12x" + "0" = "12x"
"12x3" + 15x3" = "27x3"
"12x3+4x2" + "-15x2" = "12x3-11x2"
"12x3+4x+8" + "-15x2+30x+5x-5-3" = "12x3-15x2+39x"
\end{lstlisting}

\subsubsection{PolynomialSubOperator}

Jest to bliźniacza klasa testowa dla PolynomialSumOperator. Jedyną zmianą jest to, że w tym przypadku, podajemy kolejno: odjemną, odjemnik i oczekiwaną różnicę. Warto zauważyć, że odejmowanie dwóch wielomianów, podobnie jak w przypadku liczb, sprowadza się do zsumowania pierwszego z nich i wartości przeciwnej do drugiego. Przeprowadzone testy są analogiczne, jak dla operatora sumy.

\begin{lstlisting}
"1" - "20" = "-19"
"5" - "-11" = "16"
"12x" - "0" = "12x"
"12x3" - 15x3" = "-3x3"
"12x3+4x2" - "-15x2" = "12x3+19x2"
"12x3+4x+8" - "-15x2+30x+5x-5-3" = "12x3+15x2-31x+16"
\end{lstlisting}

\subsubsection{PolynomialMulOperator}

Jest klasą, której zadaniem jest, jak najdokładniej przetestować operator mnożenia wielomianów. Testy analogicznie jak w przypadku poprzednich klas, mają na celu sprawdzenie, czy dla dwóch podanych czynników, uzyskamy pożądany wynik. Operacja iloczynu wielomianów polega na przemnożeniu wszystkich wyrazów pierwszego z nich przez wyrazy drugiego. Mnożąc dwa wyrazy, mnożymy przez siebie ich wartości, zaś wykładniki sumujemy. Następnie dokonujemy redukcji wyrazów podobnych, poprzez ich dodanie, a uzyskany w ten sposób wielomian jest szukanym iloczynem.

\begin{lstlisting}
"1" * "20" = "20"
"0" * "-55" = "-55"
"-2x" * "7" = "-14x"
"12x" * "-5x" = "-60x2"
"12x3" * "15x3" = "180x6"
"12x3+4x2" * "-15x2" = "-180x5-60x4"
"12x3+4x+8" * "-15x2+35x-8" = "-180x5+420x4-156x3+20x2+248x-64"
\end{lstlisting}

\subsection{Operatory dzielenia i modulo}

\subsubsection{PolynomialDivOperator}

PolynomialDivOperator jest klasą testową, pozwalającą na przetestowanie poprawności dzielenia wielomianów. Funkcja ją weryfikująca przyjmuje trzy parametry. Są nimi dzielna, dzielnik oraz spodziewany iloraz. Podany dzielnik nie może być wielomianem zerowym, ponieważ dzielenie przez $0$, również w przypadku wielomianów, nie jest dopuszczalne. Operacja dzielenia jest, w przypadku wielomianów, najbardziej skomplikowaną operacją. Algorytm jej wykonania przypomina nieco dzielenie pisemne w przypadku liczb całkowitych. Wyraz stojący przy najwyższej potędze dzielnej, jest dzielony przez wyraz stojący przy najwyższej potędze dzielnika, a otrzymany wynik jest zapisywany. Następnie otrzymany wyraz jest mnożony przez wielomian przeciwny do dzielnika, a otrzymany rezultat sumowany z dotychczasową wartością wielomianu. Uzyskany wynik staje się nową dzielną, a sytuacja powtarza się do momentu, gdy stopień dzielnej będzie niższy, niż stopień wielomianu dzielącego. Wartości otrzymane z kolejnych dzieleń tworzą wyrazy ilorazu, o którym wiemy, że jego stopień jest równy różnicy stopni argumentów operatora dzielenia.

\begin{lstlisting}
"20" / "20x" = "0"
"20" / "-1" = "=-20"
"-21x" / "7x" = "-3"
"-12x2" / "-6x" = "2x"
"x3+x2+x" / "x" = "x2+x+1"
"-30x3-15x" / "-5x2" = "6x"
"x6-6x4-4x3+9x2+12x+4" / "x5-4x3-2x2+3x+2" = "x"
\end{lstlisting}

\subsubsection{PolynomialModOperator}

Jest klasą testową dla operatora modulo, którego obliczanie oparte jest na wykonywaniu operacji dzielenia, z tymże w tej sytuacji zwracany jest inny wielomian. W przypadku opisanym powyżej, tzn. w momencie, gdy stopień dzielnej jest większy niż stopień aktualnie przetwarzanego wielomianu, to drugi z tych wielomianów nazywany jest resztą z dzielenia. Wiemy o nim, że jego stopień jest mniejszy niż dzielnika. W szczególnym przypadku, gdy dzielnik jest podzielny przez dzielną, uzyskana reszta jest wielomianem zerowym. Warto zaznaczyć, że podobnie jak w przypadku dzielenia, w celu weryfikacji podawane są także trzy wielomiany. Ostatni z nich jest pożądanym wynikiem operacji modulo, zaś środkowy dzielnikiem, co wymusza, by jego wartość była różna od zera. Poniżej zaprezentowane zostały te same przypadki testowe jak dla dzielenia, ale w tej sytuacji wynikiem jest rezultat operacji modulo. Dodatkowo lista testów została powiększona o testy, charakteryzujące się niezerową resztą.

\begin{lstlisting}
"20" % "20x" = "20"
"20" % "-1" = "=0"
"-21x" % "7x" = "0"
"-12x2" % "-6x" = "0"
"x3+x2+x" % "x" = "0"
"-30x3-15x" % "-5x2" = "-15x"
"x6-6x4-4x3+9x2+12x+4" % "x5-4x3-2x2+3x+2" =
 = "-2(x4+x3-3x2-5x-2)"
"x3+x2+x1+11" % "x2+1" = "10"
"(x+2)^3+1" % "x+2" = "1"
"x4-4x3+6x2-7x+3" % "(x-1)^2" = "-3x+2"
\end{lstlisting}

\subsection{PolynomialDerivative}

Zadaniem tej klasy jest sprawdzenie, czy metoda pozwalająca na obliczanie pochodnej wielomianu działa poprawnie. Jest ona wykonywana na konkretnej instancji obiektu typu Polynomial i zwraca referencję do jego pochodnej. Funkcja bazuje na operatorach dodawania i mnożenia. Ich poprawne działanie jest zatem niezbędne, by metoda ta mogła poprawnie funkcjonować.
Weryfikacja poprawności wyników następuje poprzez porównanie oczekiwanej wartości pochodnej wielomianu z wartością obliczoną. Poniżej prezentuję wybrane przypadki testowe dla tej funkcjonalności.

\begin{lstlisting}
(5)' = 0
(2a)' = 2
(a2)' = 2a
(7a3)' = 21a2
(120a130)' = 15600a129
(-a101+2a17+a4+a)' = -101a100+34a16+4a3+1
\end{lstlisting}

\subsection{PolynomialAfterElimination}

Jest to klasa weryfikująca, czy wartość wielomianu, po eliminacji pierwiastków wielokrotnych, jest poprawna. Metoda ta jest wykonywana dla danego obiektu typu Polynomial i zwraca referencję do wielomianu wynikowego. Jej wartość nie jest jednak bezpośrednio porównywana z oczekiwanym wynikiem. Wcześniej jest bowiem wykonywana jeszcze funkcja normalizująca uzyskany wielomian. Jest to niezbędne, gdyż w przeciwnym razie, wykonywane porównanie mogłoby wskazać fałszywy rezultat. Jest to spowodowane tym, że funkcja porównująca, ze względów wydajnościowych, nie normalizuje danych wielomianów. Decyzja taka była umotywowana faktem, że w takim przypadku porównanie całkowicie odmiennych wielomianów byłoby liniowe ze względu na ich stopień, zamiast odbywać się w czasie stałym. Aplikacja zakłada, że w każdym miejscu, gdzie jest to wymagane, będzie następowało jawne wywołanie odpowiedniej funkcji.

Aspektem bardzo ułatwiającym testowanie tej funkcjonalności jest możliwość podawania wielomianu wejściowego w postaci iloczynu wielu czynników. Pozwala to podawać kolejne czynniki, zawierające dane pierwiastki wielomianu, z określonymi krotnościami i spodziewać się dokładnie tych samych pierwiastków, ale jednokrotnych. Dzięki temu jest się niezależnym od innych programów, pozwalających na obliczanie zer wielomianów, ponieważ spodziewane wartości, zawarte są pośrednio w wielomianie wejściowym. Poniżej zamieszczam przykładowe testy, z uwzględnieniem podawania wejścia, w różnej postaci.

\begin{lstlisting}
"20" -> "1"
"2a" -> "a"
"(a-1)^2" -> "a-1"
"(a+3)^4" -> "a+3"
"(a+1)(a+2)" -> "(a+1)(a+2)"
"a+3)^2*(a+1)" -> "(a+3)(a+1)"
"a3*(a+2)^10" -> "a(a+2)"
"(x-1)^2*(x+1)^2*(x-2)^2" -> "(x-1)(x+1)(x-2)"
"(x+1)^4*(x2+x+1)" -> "(x+1)(x2+x+1)"
\end{lstlisting}

\subsection{PolynomialValue}

Jest klasą weryfikującą, czy wartość wielomianu w danym punkcie jest obliczana poprawnie. Funkcja weryfikująca przyjmuje dwa argumenty wejściowe – wielomian oraz dany punkt. Otrzymany wynik porównuje z trzecim parametrem, którym jest oczekiwany rezultat. Poza sprawdzeniem poprawności działania samego wielomianu, weryfikowane jest także działanie klasy Number. Ważnym aspektem jest tutaj, przede wszystkim, sprawdzenie zachowania funkcji, w przypadku dużych liczb, które często pojawiają się w momencie obliczania wartości wielomianów wysokich stopni. Należy wziąć pod uwagę, że obliczenie wartości w punkcie $x_0=10$, dla wielomianu stopnia setnego wymaga liczby posiadającej aż $101$ cyfr. To właśnie z tego względu w projekcie niezbędna była biblioteka mpir, pozwalająca na posługiwanie się takimi wartościami liczbowymi. Poniżej prezentuję przykładowe testy na obliczanie wartości wielomianu w danym punkcie.

\begin{lstlisting}
W(x) = "20", x0 = 0 -> W(0) = 20
W(x) = "a", x0 = 13 -> W(13) = 13
W(x) = "12a2", x0 = 3 -> W(3) = 108
W(x) = "(a+1)(a+2)", x0 = 3 -> W(3) = 20
W(x) = "(a+3)^2*(a+1)", x0 = -2 -> W(-2) = -1
W(x) = "(x+1)^4*(x-2)^2", x0 = 2 -> W(2) = 0
W(x) = "(x-1)^2*(x+1)^4*(x-2)^2", x0 = 3 -> W(3) = 1024
W(x) = "(x+1)^4*(x2+x+1)", x0 = 2 -> W(2) = 567
W(x) = "a39+16a+a8", x0 = 2 -> W(0) = 0x8000010100
\end{lstlisting}

\subsection{PolynomialNumberOfRoots}

Jest to klasa, sprawdzająca ile pierwiastków rzeczywistych znajduje się w podanym przedziale. Pod uwagę brane są wyłącznie różne pierwiastki, dlatego na początku funkcji weryfikującej wywoływana jest funkcja PolynomialAfterEliminationOfMultipleRoots, a dopiero na jej znormalizowanym rezultacie wykonywana jest właściwa funkcja. Jej rezultat porównywany jest ze spodziewaną wartością, podawaną w postaci zmiennej typu int. Domyślnymi granicami przedziału są wartości niewłaściwe $-\infty$ i $+\infty$, więc gdy interesują nas pierwiastki w całym zbiorze liczb rzeczywistych, nie ma potrzeby ich podawać. W przeciwnym razie są one nadpisywane wartościami pól klasy – $a$ i $b$, oznaczających lewy i prawy kraniec przedziału. Możliwe jest także sprecyzowanie granicy przedziału tylko z jednej strony, np.\ gdy szukamy liczby pierwiastków dodatnich. Poniżej zamieszczam przykładowe testy dla różnych przedziałów wyszukiwania.

\begin{lstlisting}
W(x) = "20" -> 0
W(x) = "2a" -> 1
W(x) = "-a3" -> 1
W(x) = "(a-1)^2" -> 1
W(x) = "(a+1)(a+2)" -> 2
W(x) = "(a+3)^2*(a+1)^3" -> 2
W(x) = "(x-1)^2*(x+1)^2*(x-2)^2" -> 3
W(x) = "(x+1)^4*(x2+x+1)" -> 1
W(x) = "12a2", a = -1 -> 1
W(x) = "a3*(a+2)^10", a = -1, b = 4 -> 1
W(x) = "(x+1)^4*(x-2)^2", a = 0, b = 1 -> 0
\end{lstlisting}

\subsection{PolynomialRoots}

Na początku funkcji weryfikującej dokonywana jest eliminacja pierwiastków wielokrotnych. Otrzymany wielomian poddawany jest normalizacji, a wówczas znajdowane są jego pierwiastki w podanym przedziale. Podobnie jak w przypadku klasy PolynomialNumberOfRoots, domyślnym przedziałem jest zakres od $-\infty$ do $+\infty$. Pierwiastki w wektorze wyjściowym funkcji znajdują się w kolejności ich znalezienia. Także w tym przypadku, konieczne jest więc ich posortowanie.

Poniżej zamieszczam większość testów, sprawdzających ostateczną funkcjonalność programu. Lista ta jest bardziej rozbudowana niż poprzednie, ponieważ uznałem, że jest ona zdecydowanie najistotniejszą częścią testów funkcjonalnych mojego programu.

\begin{lstlisting}
W(x) = "20" -> {}
W(x) = "2a" -> {0}
W(x) = "-a3" -> {0}
W(x) = "(a-1)^2" -> {0}
W(x) = "(a+1)(a+2)" -> {-2, -1}
W(x) = "(a+3)^2*(a+1)^3" -> {-3, -1}
W(x) = "(x-1)^2*(x+1)^2*(x-2)^2" -> {-1, 1, 2}
W(x) = "(x+1)^4*(x2+x+1)" -> {-1}
W(x) = "12a2", a = -1 -> {0}
W(x) = "a3*(a+2)^10", a = -1, b = 4 -> {0}
W(x) = "(x+1)^4*(x-2)^2", a = 0, b = 1 -> {}
\end{lstlisting}