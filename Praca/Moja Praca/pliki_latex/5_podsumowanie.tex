\chapter{Podsumowanie}

Celem pracy była implementacja algorytmu, pozwalającego na obliczanie pierwiastków rzeczywistych dowolnego wielomianu. Dokonałem tego, korzystając z twierdzenia Sturma. Przeprowadziłem analizę, jaka repzentacja wielomanu jest bardziej wydajna. Dodatkowo, porównałem także możliwości stwrzonej przez mnie aplikacji z serwisem Wolfram Alpha, który umożliwia m.\ in.\ obliczanie pierwiastków rzeczywistych wielomianów.

Stworzona przeze mnie aplikacja została podzielona na trzym moduły. Są to biblioteka statyczna, framework testowy oraz aplikacja konsola. Pierwsza z nich zawiera całą logikę aplikacji, pozwala wykonywać działania na wielomianach, korzystając z biblioteki mpir. Ta ostatnia umożliwia działania na dużych liczbach o wysokiej precyzji, przy pomocy zoptymalizowanego kodu asemblerowego, co sprawia, że przeprowadzone operacja są maksymalnie wydajne. Drugą częśc stanowi framework testowy, którzy w narzedziu Microsoft Visual Studio 2015 pozwala na definiowane i egzekucję testów jednostkowych. W moim przypadku poza prostymi testami zostały w nich zdefinionowane także bardziej skomplikowane testy funkcjonalne. Trzecim modułem jest aplikacja konsolowa, dzięki której użytkownik może wprowadzać wielomiany oraz przedział, w którym chcemy znaleźć wszystkie pierwiastki rzeczywiste z podaną precyzją.

Implementacja wielomianu została oparta na klasie abstrakcyjnej Polynomial. Na jej podstawie zostały wykonane dwie klasy, różniące się reprezentacją wielomianu na której bazują. Pierwszą z nich została klasa oparta na mapie, co pozwalało na posiadanie informacji wyłącznie o niezerowych współczynnikach. Z kolei drugą z nich jest tablica, wymuszająca reprezentację także zerowych współczynników wielomianów.

Przeprowadziłem testy wydajnościowe porównujące czas działania algorytmu dla obu klas. TODO

Dokonałem porównania możliwości wytworzonej przeze mnie aplikacji, z innym programem, który umozliwia znajdowanie pierwiastków wielomianów. Wybrałem serwis Wolfram Alpha. Porównałem interfejsy i możliwości obu aplikacji. Z przedstawionego zestawienia wynikało, że różnice są niewielkie, ale serwis Wolfram Alpha posiada kilka dodatkowych funkcji takich jak rysowanie wykresu, czy znajdowanie pierwiastków zespolonych, których nie posiada moja aplikacja.

Po dogłębnej analizie okazało się, że dla pewnych wielomianów Wolfram Alpha dokonuje zaokrągleń, które powodują pewne niedokładności w wartości znalezionych pierwiastków lub zupełnie umożliwiają ich znalezienie. Dzieje się tak, gdyż wartości współczynników wielomianów są w pewnych okolicznościach zaokrąglane. Fakt ten pokazuje, jak ważnym aspektem pracy było znalezienie sposóbu na reprezentację wartości liczbowych, dowolnej precyzji. Bez tego elementu skuteczne znajdowanie pierwiastków wielomianów jest zazwyczaj niemożliwe i działa tylko dla bardzo prostych przypadków.
