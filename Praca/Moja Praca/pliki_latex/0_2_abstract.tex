\chapter*{Abstract}

The purpose of this thesis is presentation and discussion about the way allowing finding real roots of the polynomial in examined range. The most important aspect was a choice of the appropriate method to be effective in each case and not only under certain conditions. An example of such way is the use of Sturm’s theorem. Before using it, it is necessary to eliminate of multiple roots of the polynomial, then be able to define the Sturm’s sequence. Based on it, we are able to determine the number of distinct real roots located in an interval. By theorem it is equal to the difference changes sign at both ends. The test interval can be so arbitrarily reduce to indicate the value of searched roots with the required precision.

The important aspect of this thesis is a console application, made in C++, allowing finding roots of the polynomial, which is entered by a user. For its implementation it is necessary to implement actions on polynomials and handling large numbers of high-value precision. For this purpose, the program uses the MPIR library. To ensure the quality of the program, was created a separate application module, which includes the relevant functional tests.

An interesting issue is to compare the different types of structures, allowing a polynomial representation. The first of them is the array, which are kept all the coefficients of the polynomial. Another one is a map, as an example of structure with a lack of direct access to all of its elements, which allows you to have information only about the non-zero coefficients. They were performed tests comparing the operating efficiency of both structures. The results of both of them are compared with both the theoretical values as well as with each other. In addition, the analysis was conducted, which of the structures is better for different types of polynomials.

\vspace{12pt}
\noindent\textbf{Keywords: }\\
polynomial, roots of polynomial, Sturm’s theorem, mpir library

\vspace{12pt}
\noindent\textbf{Field of science and technology in accordance with OECD requirements:}\\ 
TODO	