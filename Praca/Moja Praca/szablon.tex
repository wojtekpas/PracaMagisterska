\documentclass[twoside,a4paper]{book}

\usepackage[pdftex]{graphicx}
\usepackage{amsmath}
\DeclareMathOperator{\sign}{sign}
\usepackage{amssymb}
\usepackage{textcomp}
\usepackage[utf8]{inputenc}
\usepackage[polish]{babel}
\usepackage[T1]{fontenc}
\usepackage{array}
% pakiet stosowany do url'i w bibliografii, zamienia odnośniki na ładnie sformatowane
\usepackage{url}
% pakiety służące do numerowania i tworzenia algorytmów
\usepackage{algorithmic}
\usepackage{algorithm}
% redefinicja etykiety nagłówkowej listy algorytmów, domyślna jest po angielsku
\renewcommand{\listalgorithmname}{Spis algorytmów}

% pakiet do wyliczania skali, przydatny przy dużych obrazkach
\usepackage{pgf}
% pakiet służący do automatycznego sortowania odnośników do bibliografii
\usepackage[sort]{natbib}
% tworzenie listingów
\usepackage{listings}
% tworzenie figur wewnątrz figur
\usepackage{subfig}
% do automatycznego skracania nazw rozdziałów i podrozdziałów używanych w nagłówkach strony by mieściły się w jednej linii
\usepackage[fit]{truncate}
% fancyhdr - ładne nagłówki, definicja wyglądu nagłówka, numery stron będą umieszczane w nagłówku po odpowiedniej stronie
\usepackage{fancyhdr}
\pagestyle{fancy}
\renewcommand{\chaptermark}[1]{\markboth{#1}{}}
\renewcommand{\sectionmark}[1]{\markright{\thesection\ #1}}
\fancyhf{}
\fancyhead[LE,RO]{\bfseries\thepage}
% tutaj ograniczamy szerokość pola w nagłówku zawierającego nazwę rozdziału/podrozdziału do 95% szerokości strony
% redefinicja sposobu prezentacji nazw domyślnie wypisywanych wielkimi literami (np. domyślnie w nagłówku Spis treści będzie miał postać SPIS TREŚCI)
% Uwaga! to może popsuć wielkie litery w ogóle! Jak coś nie działa należy usunąć \nouppercase{} z poniższych definicji
\fancyhead[LO]{\nouppercase{\bfseries{\truncate{.95\headwidth}{\rightmark}}}}
\fancyhead[RE]{\nouppercase{\bfseries{\truncate{.95\headwidth}{\leftmark}}}}
\renewcommand{\headrulewidth}{0.5pt}
\renewcommand{\footrulewidth}{0pt}

% definicja typu prostego wymagana przez pierwsze strony rozdziałów itp.
% powyższe reguły niestety tych stron nie dotyczą, gdyż Latex automatycznie przełącza je pomiędzy fancy a plain
% w tym wypadku eliminujemy nagłówki i stopki na stronach początkowych
\fancypagestyle{plain}{%
 \fancyhead{}
 \fancyfoot{}
 \renewcommand{\headrulewidth}{0pt}
 \renewcommand{\footrulewidth}{0pt}
}

\parskip 0.05in
\linespread{1.3}

% makro umożliwiające otaczanie symboli okręgami
\usepackage{tikz}
% brak justowania tekstu (bazą okręgu będzie linia tekstu)
\newcommand*\mycirc[1]{%
  \begin{tikzpicture}
    \node[draw,circle,inner sep=1pt] {#1};
  \end{tikzpicture}}

% pionowe justowanie tekstu, środek okręgu pokrywa się ze środkiem tekstu
\newcommand*\mycircalign[1]{%
  \begin{tikzpicture}[baseline=(C.base)]
    \node[draw,circle,inner sep=1pt](C) {#1};
  \end{tikzpicture}}

% zmiana nazwy twierdzeń i lematów
\newtheorem{theorem}{Twierdzenie}
\newtheorem{definition}{Definicja}
\newtheorem{example}{Przykład}

% tworzenie definicji dowodu
\newenvironment{proof}[1][Dowód]{\begin{trivlist}
\item[\hskip \labelsep {\bfseries #1}]}{\end{trivlist}}
% \newenvironment{definition}[1][Definicja]{\begin{trivlist}
% \item[\hskip \labelsep {\bfseries #1}]}{\end{trivlist}}
% \newenvironment{example}[1][Przykład]{\begin{trivlist}
% \item[\hskip \labelsep {\bfseries #1}]}{\end{trivlist}}
\newenvironment{remark}[1][Wniosek]{\begin{trivlist}
\item[\hskip \labelsep {\bfseries #1}]}{\end{trivlist}}

% definicja czarnego prostokąta zwyczajowo dodawanego na koniec dowodu
\newcommand{\qed}{\nobreak \ifvmode \relax \else
      \ifdim\lastskip<1.5em \hskip-\lastskip
      \hskip1.5em plus0em minus0.5em \fi \nobreak
      \vrule height0.75em width0.5em depth0.25em\fi}

% poniższymi instrukcjami można sterować co ma być numerowane a co nie i co ma być wyświetlane w spisie treści
% \setcounter{secnumdepth}{3}
% \setcounter{tocdepth}{5}

% definicja czcionki mniejszej niż tiny (domyślnie takiej małej nie ma)
\usepackage{lmodern}
\makeatletter
  \newcommand\tinyv{\@setfontsize\tinyv{4pt}{6}}
\makeatother

% definicja jeszcze mniejszej czcionki
\usepackage{lmodern}
\makeatletter
  \newcommand\tinyvv{\@setfontsize\tinyvv{3.5pt}{6}}
\makeatother

% pakiet do obsługi wieloicowych tabel
\usepackage{longtable}
\setlength{\LTcapwidth}{\textwidth}

\usepackage[section] {placeins}

\usepackage{multirow}

\usepackage{slantsc}

% nazwa pliku ze stroną tytułową
% % allows useg @ as a @ not as special character
% required for macro redefinition
\makeatletter

% parameters definition
% they cannot conflict with other
% like bibteh attributes etc.
\def\promotor#1{\def\@promotor{#1}}
\def\promotordpt#1{\def\@promotordept{#1}}
\def\promotoruniv#1{\def\@promotoruniv{#1}}
\def\miasto#1{\def\@miasto{#1}}

\def\maketitle{
  %removal of header
  \thispagestyle{empty}%
  %changing margins to match department requirements
%   \changepage{+60pt}{+70pt}{}{-30pt}{}{-30pt}{-12pt}{-18pt}{}
  \begin{center}
    \begin{tabular}{@{}lcr@{}}
      \multirow{4}{*}{\includegraphics[height=1.9cm]{img/logo_pg.png}} &
      \LARGE{\textbf{POLITECHNIKA GDAŃSKA}} &
      \multirow{4}{*}{\includegraphics[height=1.9cm]{img/logo_eti.png}}\\
      & & \\
      &\LARGE{\textbf{Wydział Elektroniki,}}&\\
      &\LARGE{\textbf{Telekomunikacji i Informatyki}}&
    \end{tabular}
  \end{center}
  \rule{\linewidth}{0.1mm}
  \vspace{2.82cm}
  \begin{center}
    \huge{\textbf{\@author}}
  \end{center}
  \vspace{0.5cm}
  \begin{center}
    \Huge{\textbf{\@title}}
  \end{center}
  \vspace{0.5cm}
  \begin{center}
    \huge{Rozprawa doktorska}
  \end{center}
  \vspace{1.41cm}
% \renewcommand{\arraystretch}{0.85}
  \begin{tabular}{@{}p{0.46\textwidth}@{}@{}p{0.54\textwidth}@{}}
\noalign{\smallskip}
    & \hspace{-20pt}\LARGE{Promotor:}\\
\noalign{\smallskip}
\noalign{\smallskip}
    & \Large{\@promotor}\\
\noalign{\smallskip}
\noalign{\smallskip}
    & \Large{\@promotordept}\\
\noalign{\smallskip}
\noalign{\smallskip}
    & \Large{\@promotoruniv}
  \end{tabular}
  \vfill
  \begin{center}
    \Large{\@miasto, \@date}
  \end{center}
  %
%   \cleardoublepage
  % restore page defaults
%   \changepage{-60pt}{-70pt}{}{+30pt}{}{+30pt}{+12pt}{+18pt}{}
}

%restore @ sign
\makeatother

\cleardoublepage
% allows useg @ as a @ not as special character
% required for macro redefinition
\makeatletter

% parameters definition
% they cannot conflict with other
% like bibteh attributes etc.
\def\promotor#1{\def\@promotor{#1}}
\def\miasto#1{\def\@miasto{#1}}
\def\studies#1{\def\@studies{#1}}
\def\descr#1{\def\@descr{#1}}
\def\indeks#1{\def\@indeks{#1}}
\def\dept#1{\def\@dept{#1}}

\def\maketitle{
  %removal of header
  \thispagestyle{empty}%

  \begin{center}
    \begin{tabular}{lcl}
      \multirow{4}{*}{\includegraphics[height=2.5cm]{img/logo_pg.png}} &
      \textsc{\textbf{Politechnika Gdańska}} &
      \multirow{4}{*}{\includegraphics[height=2.5cm]{img/logo_eti.png}}\\
      & & \\
      &\textsc{\textbf{Wydział Elektroniki,}}&\\
      &\textsc{\textbf{Telekomunikacji i Informatyki}}&
    \end{tabular}
  \end{center}
  \vspace{1cm}
  \begin{tabular*}{\textwidth}{p{0.5\textwidth}p{0.5\textwidth}}
    \textbf{Katedra:} & \@dept\\
    &\\
    \textbf{Imię i nazwisko dyplomanta:} & \@author\\
    &\\
    \textbf{Nr albumu:} & \@indeks\\
    &\\
    \textbf{Forma i poziom studiów:} & \@studies\\
    &\\
    \textbf{Kierunek studiów:} & Informatyka\\
  \end{tabular*}
  \begin{center}
    \vspace{1cm}
    \Large{\textbf{Praca dyplomowa magisterska}}
  \end{center}
  \vspace{1cm}
  \begin{tabular*}{\textwidth}{p{\textwidth}}
    \textbf{Temat pracy:} \\ \@title\\
    \\
    \textbf{Kierujący pracą:} \\ \@promotor\\
    \\
    \textbf{Zakres pracy:} \\ \@descr\\
  \end{tabular*}
  \vspace*{\stretch{6}}
  \begin{center}
    \@miasto, \@date
  \end{center}

}

%restore @ sign
\makeatother

\cleardoublepage

% parametry strony tytułowej, zdefiniowane są w plikach z poszczególnymi stronami
% tytuł pracy
\title{Obliczanie zer wielomianów}
% autor
\author{Wojciech Pasternak}
% rok wydania
\date{2016}
% miasto, gdzie napisano pracę
\miasto{Gdańsk}
% promotor
\promotor{dr hab. inż. Robert Janczewski}
% wydział promotora, tylko dla phd_titlepage
% \promotordpt{Wydział Elektroniki, Telekomunikacji i~Informatyki}
% uczelnia promotora, tylko dla phd_titlepage
% \promotoruniv{Politechnika Gdańska}

% rodzaj studiów, tylko dla mgr_titlepage
\studies{Stacjonarne jednolite studia magisterskie}
% opis pracy, tylko dla mgr_titlepage
\descr{Obliczanie pierwiastków wielomianów i porównanie służących do tego struktur}
% nr indeksu, tylko dla mgr_titlepage
\indeks{137361}
% katedra, tylko dla mgr_titlepage
\dept{Architektury Systemów Komputerowych}

% korekta marginesów - domyślnie latex ma jakieś kosmiczne
\usepackage{anysize}
\marginsize{3.5cm}{2.5cm}{2.5cm}{2.5cm}
% po zmianie marginesów konieczne jest wymuszenie przeliczenia nagłówków
\fancyhfoffset[E,O]{0pt}

\begin{document}
% sekcja wstępna książki, numerowana rzymskimi
\frontmatter
% generacja strony tytułowej załączonej wcześniej
\maketitle
% spis treści
\tableofcontents

% właściwa część książki, numerowana arabskimi od 1
\mainmatter

\chapter{Wstęp}
\chapter{Przegląd literatury}
\section{Wielomiany}
\subsection{Definicja}

\begin{definition}
	Wielomianem zmiennej rzeczywistej x nazywamy wyrażenie:
	\begin{equation}
		\begin{split}
			&W(x) = a_0x^n + a_1x^{n-1} + a_2x^{n-2}+ ... + a_{n-1}x + a_n, \\
			&gdzie\ a_0, a_1, a_2, ..., a_{n-1}, a_n\in R, n \in N 
		\end{split}
	\end{equation}
\end{definition}

Liczby $a_0, a_1, a_2, ..., 1_{n-1}, a_n$ nazywamy współczynnikami wielomianu, natomiast n nazywamy stopniem wielomianu.
	
Szczególnym przypadkiem wielomianu jest jednomian. 

\begin{definition}
	Jednomianem zmiennej rzeczywistej x nazywamy wielomian, który posiada co najwyżej jeden wyraz niezerowy i określamy wzorem:
	\begin{equation}
		\begin{split}
			W(x) = ax^n, gdzie\ a\in R, n \in N 
		\end{split}
	\end{equation}
\end{definition}

Można, więc rozumieć wielomian jako skończoną sumę jednomianów.
Jednomianem stopnia zerowego jest stała, pojedyncza liczba rzeczywista, która w szczególności może być zerem.

\begin{definition}
	Wielomianem zerowym nazywamy, wielomian wyrażony wzorem:
	\begin{equation}
		W(x) = 0
	\end{equation}
\end{definition}

W dalszej części, jeżeli nie zaznaczymy inaczej, mówiąc wielomian, będziemy mieli na myśli pewien wielomian, nie będący wielomianem zerowym.

\subsection{Podstawowe działania na wielomianach}

Na wielomianach, tak jak na liczbach możemy wykonywać podstawowe działania. Należą do nich: porównywanie, dodawanie, odejmowanie, mnożenie, dzielenie, a także obliczanie reszty z dzielenia oraz NWD (największego wspólnego dzielnika). Jako, że wielomian zmiennej x możemy traktować jak funkcję jednej zmiennej, możemy także policzyć z niego pochodne.

\subsubsection{Porównywanie wielomianów}

Porównywanie należy do najbardziej elementarnych działań na wielomianach. Wymaga ono zwykłego porównania kolejnych współczynników, a jego długość trwania, zależy od ich liczby. Zapoznajmy się z twierdzeniem dotyczącym operacji porównywania wielomianów.

\begin{theorem}
	Dwa wielomiany uważamy za równe wtedy i tylko wtedy, gdy są tego samego stopnia, a ich kolejne współczynniki są równe.
\end{theorem}

Powyższe twierdzenie nie jest złożone, nie mniej w celu pełnego zrozumienia, zilustrujmy je przykładem. 

\begin{example}
	Mamy dane wielomian $W_1$ oraz wielomian $W_2$.
	\begin{equation}
		\begin{split}
			&W_1(x) = a_0x^n + a_1x^{n-1} + ... + a_{n-1}x + a_n \\
			&W_2(x) = b_0x^n + b_1x^{n-1} + ... + b_{n-1}x + b_n \\
		\end{split}
	\end{equation}
	Wielomiany $W_1$ oraz $W_2 $ są równe wtedy i tylko wtedy gdy
	$\forall{i\in N}\ a_i = b_i$.
\end{example}

Można zauwazyć, potencalny wpływ reprezentacji wielomianu na szybkość operacji porównania. Gdy mamy do czynienia z wielomianem, w którym uwzględniamy każdy współczynnik, także gdy jest on zerowy, złożoność czasowa porównania jest liniowa względem stopnia wielomianu. Natomiast w przypadku, gdy pomijamy wszystkie zerowe współczynniki wielomianu, złożoność również jest liniowa, ale tym razem względem liczby niezerowych współczynników wielomianów. Jak widać, w sytuacji, gdy stopień wielomianu jest znacznie większy od liczby zerowych współczynników, reprezentacja wielomianu ma niebagatelne znaczenie. \newline
Dodatkowo, podobnie jak w przypadku porównywania liczb i sprawdzania kolejnych bitów, operacja porównania kończy się w momencie stwierdzenie, że porównywane współczynniki są różne lub porównaliśmy ze sobą już wszystkie współczynniki. Wynika z tego, że zakładając stały czas porównywania dwóch liczb, będącymi współczynnikami wielomianów, operacja porównania różnych wielomianów nigdy nie jest dłuższa od stwierdzenia, że porównywane wielomiany są równe.

\subsubsection{Suma wielomianów}

Dodawanie to kolejne elementarne działanie na wielomianach, które nie wymaga wykonywania skomplikowanych obliczeń.

\begin{theorem}
	Aby dodać dwa wielomiany, należy dodać ich wyrazy podobne.
\end{theorem}

Podobnie jak w przypadku porównywania czas dodawania wielomianów jest liniony, a ich reprezentacja ma zasadniczy wpływ na liczbę operacji dodawania. Pokażmy zastosowanie powyższego twierdzenia na przykładzie.

\begin{example}
	Mamy dane wielomian $W_1$ oraz wielomian $W_2$.
	\begin{equation}
		\begin{split}
			&W_1(x) = a_0x^n + a_1x^{n-1} + ... + a_{n-1}x + a_n \\
			&W_2(x) = b_0x^n + b_1x^{n-1} + ... + b_{n-1}x + b_n \\
		\end{split}
	\end{equation}
	Zdefiniujmy trzeci wielomian: $W_3(x) = W_1(x) + W_2(x)$. Wówczas:
	\begin{equation}
		W_3(x) = (a_0+b_0)x^n + (a_1+b_1)x^{n-1} + ... + (a_{n-1} + b_{n-1})x + a_n + b_n
	\end{equation}
\end{example}

Na powyższym przykładzie łatwo zaobserwować, że stopień sumy dwóch wielomianów nie może być większy od większego ze stopni dodawanych wielomianów. Znajduje to potwierdzenie w twierdzeniu, dotyczącym stopnia sumy wielomianów.

\begin{theorem}
	\begin{equation}
		\deg(W_1 + W_2) \le max(deg(W_1),\ deg (W_2))
	\end{equation}
\end{theorem}

W przedstawionym twierdzeniu, należy uwagę na operator mniejsze równe. W przypadku gdy oba te wielomiany są tego samego stopnia, o przeciwnym współczynniku przy najwyższej potędze, to stopień ten będzie mniejszy.

\subsubsection{Różnica wielomianów}

Odejmowanie to operacja bliźniacza do dodowania, nie tylko w przypadku liczb, ale także w przypadku wielomianów. By pokazać olbrzymie podobieństwo tych operacji, zacznijmy od zapoznania się z definicją wielomianu przeciwnego.

\begin{definition}
	Wielomianem przeciwnym nazywamy wielomian, którego wszystkie współczynniki są przeciwne do danych.
\end{definition}

Spójrzmy na poniższy przykład, pokazujący, że dla każdego wielomianu można bardzo prosto zdefiniować wielomian przeciwny, zmieniając znak wszystkich jego współczynników.

\begin{example}
	Mamy dany wielomian $W_1$.
	\begin{equation}
		W_1(x) = a_0x^n + a_1x^{n-1} + ... + a_{n-1}x + a_n
	\end{equation}
	Zdefiniujmy drugi wielomian: $W_2(x) = -W_1(x)$. Wówczas:
	\begin{equation}
		W_1(x) = -a_0x^n + (-a_1)x^{n-1} + ... + (-a_{n-1})x + (-a_n)
	\end{equation}
\end{example}

Wiemy już, czym jest wielomian przeciwny. Przedstawmy teraz twierdzenie mówiące jak odejmować od siebie wielomiany.

\begin{theorem}
	Aby obliczyć różnicę wielomianów $W_1$ i $W_2$, należy dodać ze sobą wielomiany $W_1$ i $-W_2$, czyli wielomian przeciwny do wielomianu $W_2$.
\end{theorem}

Jak widać, przedstawione twierdzenie potwierdza analogię obliczania róznicy i sumy wielomianów. Spójrzmy na przykład, pokazujący jak obliczać różnicę wielomianów, potrafiając już je do siebie dodawać.

\begin{example}
	Mamy dane wielomian $W_1$ oraz wielomian $W_2$.
	\begin{equation}
	\begin{split}
	&W_1(x) = a_0x^n + a_1x^{n-1} + ... + a_{n-1}x + a_n \\
	&W_2(x) = b_0x^n + b_1x^{n-1} + ... + b_{n-1}x + b_n \\
	\end{split}
	\end{equation}
	Zdefiniujmy trzeci wielomian: $W_3(x) = W_1(x) - W_2(x)$. Wówczas:
	\begin{equation}
		W_3(x) = (a_0-b_0)x^n + (a_1-b_1)x^{n-1} + ... + (a_{n-1} - b_{n-1})x + a_n - b_n
	\end{equation}
\end{example}

Warto zauważyć, że wielomianem neutralnym ze względu na dodawanie i odejmowanie jest wielomian $W(x)=0$. Oznacza to, że po dodaniu lub odjęciu wielomianu neutralnego, dostaniemy wynik, będący danym wielomianem.

\subsubsection{Iloczyn wielomianów}

Mnożenie to kolejne operacja zaliczająca się do podstawowych działań na wielomianach. Jego zasady przypominają nieco zwykłe mnożenie. Dokonujemy przemnożenia odpowiednich wyrazów, z tą różnicą, że w tym przypadku po prostu dodajemy wartości potęg dla odpowiednich współczynników. Zapoznajmy się z twierdzeniem, mówiącym dokładnie jak należy obliczać iloczyn wielomianów.

\begin{theorem}
	Aby pomnożyć dwa wielomiany, należy wymnożyć przez siebie wyrazy obu wielomianów, a następnie dodać do siebie wyrazy podobne.
\end{theorem}

Jak wynika z przedstawionej definicji poza mnożeniem dwóch liczb, mnożenie wielomianów w części polega na redukcji wyrazów podobnych, czyli operacji bazującej na dodawaniu. Spójrzmy na przykład, pokazujący jak definiuje się wielomian, będacy iloczynem dwóch wielomianów.

\begin{example}
	Mamy dane wielomian $W_1$ oraz wielomian $W_2$.
	\begin{equation}
	\begin{split}
	&W_1(x) = a_0x^n + a_1x^{n-1} + ... + a_{n-1}x + a_n \\
	&W_2(x) = b_0x^m + b_1x^{m-1} + ... + b_{m-1}x + b_m \\
	\end{split}
	\end{equation}
	Zdefiniujmy trzeci wielomian: $W_3(x) = W_1(x) * W_2(x)$. Wówczas:
	\begin{equation}
		\begin{split}
		W_3(x) = &(a_0b_0)x^{n+m} + (a_0b_1+a_1b_0)x^{n+m-1} + (a_0b_2+a_1b_1+a_2b_0)x^{b+m-1} + ... \\
		&+ (a_{n-2}b_m+a_{n-1}b_{m-1}+a_nb_m)x^2 + (a_{n-1}b_m + a_nb_{m-1})x + a_nb_m \\
		\end{split}
	\end{equation}
\end{example}

Można zauważyć, że po wymnożeniu wszystkich współczynników wielomianu liczba wyrazów iloczynu wynosi $(n+1)(m+1)$. Po dokonaniu redukcji wyrazów podobnych liczba ta ulega zmniejszeniu do wartości $n+m+2$. Oznacza to zmianę liczby wyrazów z wartości kwadratowej, do wartości liniowej względem stopni wielomianów. Liczba wyrazów podobnych, po przemnożeniu dwóch wielomianów jest symetryczna względem wykładników potęg poszczególnych współczynników. Można zauważyć, że skrajne wyrazy posiadają tylko po jednym wyrazie potęgi, a zbliżając się do współczynników o środkowych indeksach, liczba ta wzrasta, aż do wartości równej połowie stopnia otrzymanego wielomianu. \newline

Widzimy, że czas operacji mnożenia wielomianów jest kwadratowy, wzgędem stopni mnożonych przez siebie czynników. Należy zauważyć, że jeżeli użyjemy reprezentacji wielomianu, w których posiadamy informację tylko o jego niezerowych współczynników, to czas operacji mnożenie będzie nadal kwadratowy, jednak względem liczby tych współczynników. Dla wielomianów wysokich stopni, w których zaledwie kilka współczynników jest niezerowych różnica ta może być niegatelna i w skrajnych przypadkach czas operacji może zmniejszyć się z kwadratowego, do czasu stałego.

Na podstawie powyższego przykładu, możemy także zaobserwować, że stopień wielomianu, będącego iloczynem dwóch wielomianów niezerowych, jest standardowo równy sumie stopni tych wielomianów. Wyjątkiem jest sytuacja gdy jeden z czynników jest wielomianem zerowym. Wówczas wynik takiej operacji również będzie wielomianem zerowym. Fakt ten znajduje potwierdzenie w poniższym twierdzeniu.

\begin{theorem}
	\begin{equation}
	\begin{split}
	&deg(W_1 * W_2) = deg(W_1) + deg(W_2),\ dla\ W_1(x) != 0, W_2(x) != 0 \\
	&W_3(x) = W_1(x) * W_2(x) = 0,\ w \ pozostałych\ przypadkach
	\end{split}
	\end{equation}
\end{theorem}

Z powyższego twierdzenia można zauważyć, że stopień otrzymanego wielomianu nigdy nie będzie wyższy od dwukrotności większego ze stopni mnożonych wielomianów.

\subsubsection{Iloraz wielomianów}

Dzielenie to zdecydowanie najtrudniejsza z elementarnych operacji na wielomianach. Aby dobrze zrozumieć jego zasady zapoznajmy się z definicją podzielności wielomianów oraz dzielnika wielomianu.

\begin{definition}
	Wielomian W(x) nazywamy podzielnym przez niezerowy wielomian P(x) wtedy i tylko wtedy, gdy istnieje taki wielomian Q(x), że spełniony jest warunek $W(x) = P(x) * Q(x)$. Wówczas: wielomian Q(x) nazywamy ilorazem wielomianu W(x) przez P(x), zaś wielomian P(x) nazywamy dzielnikiem wielomianu W(x).
\end{definition}

Bardzo ważnym aspektem obliczania ilorazu wielomianów jest reszta z dzielenia. Spójrzmy na poniższą definicję.

\begin{definition}
	Wielomian R(x) nazywamy resztą z dzielenia wielomianu W(x) przez niezerowy wielomian P(x) wtedy i tylko wtedy, gdy istnieje taki wielomian Q(x), że spełniony jest warunek $W(x) = P(x) * Q(x) + R(x)$.
\end{definition}

Łatwo zauważyć analogię w wyżej przedstawionych wzorach. Różnią się one właśnie wielomianem $R(x)$, czyli resztą z dzielenia. Gdy jest ona wielomianem zerowym, to znaczy, że mamy do czynienia z dzieleniem bez reszty i mówimy o podzielności dwóch wielomianów. Spórzmy na przykład, w którym zdefiniowane zostały dwa wielomiany, będące ilorazem i resztą z dzielenia dwóch wielomianów.

\begin{example}
	Mamy dane wielomian W oraz wielomian P.
	\begin{equation}
		\begin{split}
			&W(x) = a_0x^n + a_1x^{n-1} + ... + a_{n-1}x + a_n \\
			&P(x) = b_0x^m + b_1x^{m-1} + ... + b_{m-1}x + b_m \\
		\end{split}
	\end{equation}
	Zdefiniujmy wielomian: $Q(x) = \frac{W(x)}{P(x)}$ oraz $Q(x) = W(x)\ mod\ P(x)$. Wówczas:
	\begin{equation}
		\begin{split}
			&Q(x) = c_0x^{n-m} + c_1x^{n-m-1} + ... + c_{n-m-1}x + c_{n-m},\ gdzie\ c_0\ne 0 \\
			&R(x) = d_0x^{n-m-1} + d_1x^{n-m-2} + ... + d_{n-m-2}x + d_{n-m-1} \\
		\end{split}
	\end{equation}
\end{example}

Zwróćmy uwagę na potęgi stojące przy najwyższych potęgach wielomianów $Q(x)$ oraz $R(x)$. Widzimy, że stopień ilorazu wielomianów jest zawsze równy różnicy stopni wielomianów, będących dzielną i dzielnikiem. Najważniejszym aspektem jest jednak fakt, że stopień reszty z dzielenia wielomianów jest zawsze mniejszy od stopnia ilorazu. Nie można natomiast ustalić jego wartości, bez dokładnej znajomości wielomianów W i P. W przykładzie podkreślony został fakt, że współczynnik stojący przy x, o potędze $n-m-1$ może być zerem. To samo tyczy się także kolejnych współczynników. Gdy wszystkie one są zerami, to znaczy, że mamy do czynienia z resztą, będąco wielomianem zerowym. Oznacza to wówczas, że wielomian W jest podzielny przez wielomian P. Poniżej znajduje się twierdzenie, o stopniach wielomianów, będących ilorazem i resztą z dzielenia.

\begin{theorem}
	\begin{equation}
	deg(W_1\ mod\ W_2) < deg(W_1 / W_2) = deg(W_1) - deg(W_2)
	\end{equation}
\end{theorem}

Jak widać, twierdzenie potwierdza nasze obserwacje i wnioski dotyczące stopni obu wielomianów.


\subsubsection{Pochodna wielomianu}

Wielomiany jako przykład funkcji ciągłej, pozwalają na obliczanie pochodnych. By przekonać się, że jest to przykład jednej z prostszych operacji na wielomianach, zapoznajmy się z definicją.

\begin{definition}
	Dany jest wielomian W, określony wzorem $W(x) = a_0x^n + a_1x^{n-1} + ... + a_{n-1}x + a_n$. Pochodną wielomianu W nazywamy wielomian W' i wyrażamy wzorem:
	\begin{equation}
	W(x) = na_0x^{n-1} + (n-1)a_1x^{n-2} + ... + 2a_{n-2}x + a_{n-1}
	\end{equation}
\end{definition}

Widzimy, że powstały wielomian powstał poprzez pomnożenie wartości każdego z współczynników przez stojącą przy danym wyrazie potęgę, a następnie obniżenie jej wartości o jeden. W ten sposób potęgi wszystkich wyrazów wielomianu obniżają się. Wyjątkiem jest tutaj potęga o wartości zero, czyli stała. Pochodna z funkcji stałej jest zawsze równa zero, dlatego została pominięta w powyższym wzorze. O stopniu pochodnej wielomianu mówi poniższe twierdzenie.

\begin{theorem}
	\begin{equation}
	\begin{split}
		&deg(W') = deg(W) - 1,\ dla\ deg(W) > 0 \\
		&W(x) = 0,\ w\ pozostałych\ przypadkach
	\end{split}
	\end{equation}
\end{theorem}

Jak widać dla wszystkich wielomianów, nie będących stałą liczbową, stopień ich pochodnej ulega zmniejszeniu o jeden. Czas operacji obliczania pochodnej wielomianu jest porównywalny, z obliczaniem sumy wielomianów, gdyż wystarczy, że dokonamy jednokrotnego obliczania każdego z współczynników.

\subsubsection{NWD wielomianów}

\subsection{Dodatkowe twierdzenia dotyczące wielomianów}

Istnieje mnóstwo twierdzień dotyczących wielomianów. Zapoznajmy się z tymi, które ułatwią nam znajdowanie pierwiastków wielomianów.

\begin{theorem}
	Jeżeli wielomian W(x) podzelimy przez dwumian $x - x_0$, to reszta z tego dzielenia jest równa wartości tego wielomianu dla $x = x_0$.
\end{theorem}

W szczególnym przypadku reszta ta może być równa 0. Oznacza to, że liczba $x_0$ jest pierwiastkiem tego wielomianu. Wynika z tego bezpośrednio kolejne twierdzenie, znane jako twierdzenie Bezout.

\begin{theorem}[Bezout]
	Liczba $x_0$ jest pierwiastkiem wielomianu W(x) wtedy i tylko wtedy, gdy wielomian jest podzielny przez dwumian $x - x_0$.
\end{theorem}

Dodatkowo zapoznajmy się jeszcze z krótkim dowodem poprawności powyższego twierdzenia.

\begin{proof}
	Jeżeli liczba $x_0$ jest pierwiastkiem wielomianu W, to wielomian ten możemy wyrazić jako iloczyn dwumianu $x - x_0$ oraz pewnego wielomianu Q: $W(x) = (x-x_0) * Q(x)$. Wyznaczając z tego wyrażenia wielomian Q, otrzymujemy $Q(x) = \frac{W(x)}{x-x_0}$. Widzimy zatem, że dzieląc wielomian W(x) przez dwumian $x-x_0$, otrzymujemy bez reszty, wielomian Q(x).
\end{proof}

Przejdźmy teraz, to twierdzenia mówiący o pierwiastkach wielokrotnych, bazującego na twierdzeniu Bezout.

\begin{theorem}
	Liczba $x_0$ jest pierwiastkiem k-krotnym wielomianu W(x) wtedy i tylko wtedy, gdy wielomian jest podzielny przez $(x - x_0)^k$ i nie jest podzielny przez $(x - x_0)^{k+1}$.
\end{theorem}

Dowód poprawności jest analogiczny, jak dla twierdzenia Bezout. Jedyną różnicą jest to, że wielomian W przedstawiamy jako: $W(x) = (x-x_0)^k * Q(x)$.

\begin{theorem}
	Każdy wielomian W(x) nie będący wielomianem zerowym jest iloczynem czynników stopnia co najwyżej drugiego
\end{theorem}

\begin{theorem}
	Niezerowy wielomian, o współczynnikach rzeczywistych, jest jednoznacznie rozkładalny na czynniki liniowe lub nierozkładalne czynniki kwadratowe, o współczynnikach rzeczywistych.
\end{theorem}

Oznacza to, że nie da się rozłożyć jednego wielomianu na czynniki, na dwa różne sposoby, tzn. tak, by istniał chociaż jeden czynnik (lub jego proporcjonalny odpowiednik), nie występujący w drugim rozkładzie. Ważnym aspektem, wynikającym z powyższego twierdzenia jest mowa, o tym że nie wszystkie wielomiany da się rozłożyć na czynniki liniowe, o współczynnikach całkowitych. Spójrzmy na pokazujący to przykład.

\begin{example}
	Mamy dany wielomian $W(x)=x^3-1$. Rózłóżmy go na czynniki.
	Wiemy, że pierwiastkiem tego wielomianu jest $x_0 = 1$, zatem możemy przedstawić wielomian W jako iloczyn wielomianu $x-1$ oraz drugiego wielomianu.
	\begin{equation}
	\begin{split}
	W(x)=&x^3-1=x^3-x^2+x^2-x+x-1=(x-1)x^2+(x-1)x+x-1 = \\
	&=(x-1)(x2+x+1) \\
	&\Delta = 1^2 - 4*1*1 = 1 - 4 = -3 < 0 - brak\ pierwiastków\ rzeczywistych\\
	\end{split}
	\end{equation}
	Jak widać drugi z czynników jest właśnie przykładem nierozkładalnego czynnika kwadratowego, o współczynnikach rzeczywistych.
\end{example}

Powyższy czynnik da się rozłożyć na dwa czynniki liniowe, o współczynnikach zespolonych. W pracy tej będziemy jednak mówić wyłącznie o współczynnikach rzeczywistych, najczęściej zawężając jeszcze zbiór potencjalnych współczynników, do liczb wymiernych. \newline
Zapoznajmy się z kolejnym twierdzeniem, mówiącym o wartości wielomianu w danym punkcie.

\begin{theorem}
	Każdy wielomian stopnia nieparzystego, ma przynajmniej jeden pierwiastek rzeczywisty.
\end{theorem}

\begin{theorem}
	Dany jest wielomian $W(x) = x^n + a_{n-1}x^{n-1} + ... + a_1x + a_0$, o współczynnikach całkowitych. Jeżeli wielomian W posiada pierwiastki całkowite, to są one dzielnikami wyrazu wolnego $a_0$.
\end{theorem}

\begin{theorem}
	Dowolny wielomian $W_1(x) = \frac{a_n}{b_n}x^n + \frac{a_{n-1}}{b_{n-1}}x^{n-1} + ... + \frac{a_1}{b_1}x + \frac{a_0}{b_0}$, o współczynnikach wymiernych, można przekształcić w wielomian $W_2(x) = k * W_1(x)$, o współczynnikach całkowitych i tych samych pierwiastkach, co wielomian W. Wówczas:
	\begin{equation}
	k = m * NWW(b_0, b_1, ..., b_{n-1}, b_n),\ gdzie\ m\in Z
	\end{equation}
\end{theorem}

\section{Eliminacja pierwiastków wielokrotnych}

\begin{theorem}
	Jeżeli liczba jest pierwiastkiem k-krotnym wielomianu W, to jest pierwiastkiem (k-1)-krotnym pochodnej tego wielomianu.
\end{theorem}

\begin{example}
	Mamy dany wielomian $W(x) = x^3 + 2x^2 + x$. Obliczmy teraz kolejne pochodne wielomianu W.
	\begin{equation}
	\begin{split}
	&W'(x) = 3x^2 + 2*2x + 1 = 3x^2 + 4x + 1 \\
	&W^{(2)}(x) = 2*3x + 4 = 6x + 4 \\
	&W^{(3)}(x) = 6
	\end{split}
	\end{equation}
	Obliczmy teraz pierwiastki wielomianu W i jego kolejnych pochodnych.
	\begin{equation}
	\begin{split}
	&W(x) = x^3 + 2x^2 + x = x(x^2 + 2x +1) = x(x + 1)^2 \\
	&x_1 = 0,\ k_1 = 1,\ x_2 = -1,\ k_2 = 2 \\
	&W'(x) = 3x^2 + 4x + 1 \\
	&\Delta = 4^2 - 4*3*1 = 16 - 12 = 4 \\
	&\sqrt{\Delta} = 2 \\
	&x_1 = \frac{-4-2}{2*3} = \frac{-6}{6} = -1,\ k_1 = 1,\ x_2 = \frac{-4+2}{2*3} = \frac{-2}{6} = -\frac{1}{3},\ k_2 = 1 \\
	&W^{(2)}(x) = 6x + 4 = 6 (x + \frac{2}{3}) \\
	&x_1 = -\frac{2}{3},\ k_1 = 1 \\
	&W^{(3)}(x) = 6\ -\ brak\ pierwiastków \\
	\end{split}
	\end{equation}
\end{example}

Jak widać powyższy przykład potwierdza zastosowanie przedstawionego twierdzenia. Widzimy, że krotność wszystkich pierwiastków ulega zmniejszeniu o 1, w kolejnej pochodnej. Dodatkowo możemy zauwazyć, że pochodna może zawierać także pierwiastki, których nie miał dany wielomian. Ma to miejsce w przypadku, gdy wielomian, posiada przynajmniej dwa różne pierwiastki. Potwierdza to poniższe twierdzenie.

\begin{theorem}
	Liczba nowych pierwiastków pochodnej wielomianu W' (takich których nie posiadał wielomian W) jest równa liczbie różnych pierwiastków wielomianu pomniejszonej o jeden.
\end{theorem}

\begin{proof}
	Zdefiniujmy wielomian: $W(x)=(x-x_1)^{k_1}*(x-x_2)^{k_2}*...*(x-x_{m-1})^{k_{m-1}}*(x-x_m)^{k_m}$. Przyjmijmy, że wielomian W jest stopnia n i posiada m różnych pierwiastków, gdzie $1\le m\le n,\ k_1+k_2+...+k_{m-1}+k_m=n$. Zdefiniujmy wielomian $P(x)=(x-x_1)*(x-x_2)*...*(x-x_{m-1})*(x-x_m)$, w którym każdy z pierwiastków wielomianu W występuję dokładnie jeden raz. Pierwiastków jest m, zatem wielomian P jest stopnia m. Dzieląc wielomian W przez wielomian P, otrzymujemy bez reszty wielomian $Q(x)=(x-x_1)^{k_1-1}*(x-x_2)^{k_2-1}*...*(x-x_{m-1})^{k_{m-1}-1}*(x-x_m)^{k_m-1}$, który posiada każdy z pierwiastków wielomianu W, krotności pomniejszonej o jeden. Skoro krotność każdego z m pierwiastków uległa zmniejszeniu o jeden, to stopień wielomianu Q w stosunku do wielomianu W zmniejszył się o m. Stopień wielomianu Q jest wieć równy n-m.
	Na mocy twierdzenia wiemy także, że wielomian Q(x) jest podzielny również przez wielomian W', stopnia n-1. Zatem wielomian W' możemy przedstawić jako iloczyn wielomianu Q oraz innego wielomianu: $W'(x)=Q(x)*V(x)$. Wielomian V, zawiera wszystkie pierwiastki W', których nie posiadał wielomian W. Łatwo policzyć, że stopień wielomianu V jest równy: $\deg(V)=deg(W')-deg(Q)=(n-1)-(n-m)=n-1-n+m=m-1$. Jak widać, właśnie udowodniliśmy, że stopień ten jest równy liczbie różnych pierwiastków wielomianu W pomniejszonej o jeden.
\end{proof}

\begin{example}
	Dany jest wielomian W, określony wzorem: $W(x)=x^6-6x^4-4x^3+9x^2+12x+4$. Dokonajmy eliminacji pierwiastków wielokrotnych dla wielomianu W. \\
	Obliczamy pochodną wielomianu.
	\begin{equation}
	\begin{split}
	&W'(x)=6*x^5-4*6x^3-3*4x^2+2*9x+12=\\
	&=6x^5-24x^3-12x^2+18x+12=6(x^5-4x^3-2x^2+3x+2)\\
	\end{split}
	\end{equation}
	Obliczmy teraz resztę z dzielenia wielomianu W przez wielomian W'.
	\begin{equation}
	\begin{split}
	&x\\\hline
	&x^6-6x^4-4x^3+9x^2+12x+4 : (x^5-4x^3-2x^2+3x+2) \\
	&-x^6+4x^4+2x^3-3x^2-2x \\\hline
	&-2x^4-2x^3+6x^2+10x \\
	\end{split}
	\end{equation}
	Kluczowa jest wartość reszty z dzielenia. Gdyby otrzymana reszta z dzielenia była wielomianem zerowym, to pochodna wielomianu było równocześnie NWD(W, W'). W tym przypadku tak jednak nie jest, więc wykonujemy analogiczną operację z tą różnicą, że nową dzielną jest dotychczasowy dzielnik a reszta z wielomianu jest nowym dzielnikiem. Tę operację wykonujemy tak długo, jak otrzymana reszta jest niezerowego stopnia. W przypadku gdy jest ona jednocześnie wielomianem zerowym, to podobnie jak wyżej, aktualny dzielnik, jest naszym NWD(W, W'). W przypadku gdy otrzymana reszta jest niezerowym wielomianem stopnia zerowego, to największym wspólnym dzielnikiem wielomianów jest pewna niezerowa stała.
	\begin{equation}
	\begin{split}
	&x\\\hline
	&x^6-6x^4-4x^3+9x^2+12x+4 : (x^5-4x^3-2x^2+3x+2) \\
	&-x^6+4x^4+2x^3-3x^2-2x \\\hline
	&-2x^4-2x^3+6x^2+10x \\
	\end{split}
	\end{equation}
\end{example}

\section{Twierdzenie Sturma}

\begin{definition}
	Ciąg Sturma
\end{definition}

\begin{theorem}
	Jeżeli wielomian W(x) nie ma pierwiastków wielokrotnych, to liczba pierwiastków rzeczywistych w przedziale $a<x\le y)$, jest równa $Z(a) - Z(b)$.
\end{theorem}

W ogólności twierdzenie Sturma można zastosować dla przedziału $(-\infty,+\infty)$, dopuszczając w ciągu Sturma wartości niewłaściwe $+\infty$ oraz $-\infty$. Wówczas $Z(-\infty)$ będzie oznaczać liczbę zmian znaków w ciągu $W(-\infty), W_1(-\infty), W_2(-\infty),..., W_m(-\infty)$, zaś $Z(+\infty)$ liczbę zmian w ciągu $W(+\infty), W_1(+\infty), W_2(+\infty),..., W_m(+\infty)$.

\begin{theorem}
	Liczba różnych pierwiastków rzeczywistych wielomianu W(x) jest równa $Z(-\infty)-Z(+\infty)$.
\end{theorem}

Stosując twierdzenie Sturma dla coraz mniejszych przedziałów możliwe jest wyznaczenie pierwiastków wielomianu z dowolną dokładnością. Sposób ten określony jest mianem metody Sturma.
Twierdzenie Sturma jest bardzo mocnym środkowiem używanym do znajdowania pierwiastków w określonym przedziale. Może być używana nie tylko dla wielomianów, ale dowolnej różniczkowalnej funkcji ciągłej.

Przeanalizujmy teraz sposób konstruowania ciągu Sturma dla wielomianu W. Pierwszym wyrazem ciagu jest sam wielomian W. Z kolei drugim wyrazem jest pochodna wielomianu W. Kolejne wyrazy ciągu Sturma wyznaczamy obliczając resztę z ilorazu dwóch poprzednich wyrazów. Dzieje się to do uzyskania pierwszej reszty wielomianu, będącą wielomianem stopnia zerowego. Za każdym razem dzieląc wielomian stopnia n, przez wielomian stopnia m, gdzie $m<n$, mamy gwarancję, że wielomian będacy resztą tego ilorazu będzie stopnia mniejszego niż m. Korzystając z tego faktu, wiemy, że liczba wyrazów ciągu Sturma dla wielomianu W jest niewiększa od $\deg(W)+1$.

\begin{definition}
	Liczbą zmian znaków ciągu Sturma dla wielomianu W(x) w punkcie x, obliczamy zliczając liczbę zmian pomiędzy kolejnymi wyrazami, pomijac te o wartości równej zero w punkcie x.
\end{definition}

\begin{theorem}
	Liczba zmian znaków ciągu Sturma dla wielomianu W(x) jest mniejsza od liczby wyrazów tego ciągu i niewiększa od stopnia wielomianu W(x).
\end{theorem}

Można, więc zauważyć, że warunkiem wystarczającym i jednocześnie koniecznym do tego by wielomian W(x) stopnia n, posiadający wyłącznie pierwiastki jednokrotne, posiadał n pierwiastków rzeczywistych jest to by wartość ciągu Sturma wynosiła n dla $x=-\infty$ oraz 0 dla $x=+\infty$. 

Warto zauwazyć, że dla liczb dostatecznie dużych, co do wartości bezwzględnej, z uwzględnieniem wartości niewłaściwych $+\infty$ oraz $-\infty$ liczba zmian znaków zależy wyłącznie od współczynnika stojącego przy najwyżej potędze wielomianu. Znajduje to potwierdzenie w poniższym twierdzeniu.

\begin{theorem}
	Jeżeli współczynnik stojący przy najwyższej potędze wielomianu jest większy od 0, to wartość tego wielomianu jest większa od zera w punkcie $x=+\infty$ oraz mniejsza od zera w punkcie $x=-\infty$.
	Z kolei, gdy współczynnik stojący przy najwyższej potędze wielomianu jest mniejszy od 0, to wartość tego wielomianu jest mniejsza od zera w punkcie $x=+\infty$ oraz większa od zera w punkcie $x=-\infty$.
\end{theorem}

Na podstawie powyższego twierdzenia można zauważyć, że nie ma potrzeby wyliczania wartości wyrazów ciągu sturma dla wartości niewłaściwych, tzn. $x=-\infty$ oraz $x=+\infty.$ Dzieje się tak dlatego, że wartość wielomianu nie jest ważna, a istotny jest jedynie znak. Fakt ten ma duży wpływ na optymalizacje obliczeń, gdyż dla wielomianów wysokich stopni ograniczamy się jedynie do sprawdzenia znaku przy najwyższej potędze, pomijąc wykonywanie potęgowania, mnożenia i sumowania kolejnych wyrazów wielomianu.
Jeżeli zależy nam, na obliczeniu wartości wielomianu, dla odpowiednio dużych x, będącego ilorazem dwóch wielomianów, możemy skorzystać z poniższego twierdzenia.

\begin{theorem}
	Dany jest wielomian $W_1(x) = a_0x^n + a_1x^{n-1} + a_2x^{n-2} + ... + a_{n-1}x  + a_n,\ gdzie \ a_0 \ne 0$, stopnia n oraz wielomian $W_2(x) = b_0x^{n-1} + b_1x^{n-2} + b_2x^{n-3} + ... + b_{n-2}x  + b_{n-1},\ gdzie\ b_0 \ne 0$, stopnia n-1. Wówczas $\lim_{x \to +\infty}\frac{W_1(x)}{W_2(x)} = \frac{a_0}{b_0}$ oraz $\lim_{x \to -\infty}\frac{W_1(x)}{W_2(x)} = -\frac{a_0}{b_0}.$
\end{theorem}

Jak widać jesteśmy w stanie ustalić znak, a nawet dokładną wartość, największego współczynnika wielomianu, będącego ilorazem dwóch innych wielomianów, wykonując proste dzielenie dwóch liczb, będących odpowiednio najwyższymi współczynnikami wielomianów - dzielnej i dzielnika.



.\\
.\\
.\\
.\\
.\\
.\\\\
.\\\\
.\\\\
.\\
.\\
.\\
.\\
.\\
.\\\\
.\\\\
.\\\\
.\\
.\\
.\\
.\\
.\\
.\\\\
.\\\\
.\\\\

\chapter{Opis Rozwiązania}
\section{Podział na moduły}
\section{Główne klasy}
\section{Główne funkcje}
\section{Zewnętrze biblioteki}
\section{Instrukcja programu}

\chapter{Przeprowadzone testy}
\section{Testy jednostkowe}
\section{Testy interfejsu użytkownika}
\section{Testy wydajności czasowej}
\section{Testy wydajności pamięciowej}
\chapter{Podsumowanie}


\backmatter


% rodzaj bibliografii
\bibliographystyle{plain}
% plik z wpisami bibliograficznymi
\nocite{barbeau2003polynomials}
\nocite{buell2004algorithmic}
\nocite{burden2015numerical}
\nocite{childs2012concrete}
\nocite{granlund2015gnu}
\nocite{kryszewski2014wyklad}
\nocite{malik2009data}
\nocite{mcnamee2007numerical}
\nocite{mora2003solving}
\nocite{pan2012structured}
\nocite{polskie1968wiadomosci}
\nocite{sierpinski1951zasady}
\nocite{Warmus_Mieczyslaw_(1918-2007)_Metody}
\bibliography{bibliografia}



\end{document}