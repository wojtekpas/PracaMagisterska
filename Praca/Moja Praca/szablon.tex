\documentclass[twoside,a4paper]{book}

\usepackage[pdftex]{graphicx}
\usepackage{amsmath}
\usepackage{amssymb}
\usepackage{textcomp}
\usepackage[utf8]{inputenc}
\usepackage[polish]{babel}
\usepackage[T1]{fontenc}
\usepackage{array}
% pakiet stosowany do url'i w bibliografii, zamienia odnośniki na ładnie sformatowane
\usepackage{url}
% pakiety służące do numerowania i tworzenia algorytmów
\usepackage{algorithmic}
\usepackage{algorithm}
% redefinicja etykiety nagłówkowej listy algorytmów, domyślna jest po angielsku
\renewcommand{\listalgorithmname}{Spis algorytmów}

% pakiet do wyliczania skali, przydatny przy dużych obrazkach
\usepackage{pgf}
% pakiet służący do automatycznego sortowania odnośników do bibliografii
\usepackage[sort]{natbib}
% tworzenie listingów
\usepackage{listings}
% tworzenie figur wewnątrz figur
\usepackage{subfig}
% do automatycznego skracania nazw rozdziałów i podrozdziałów używanych w nagłówkach strony by mieściły się w jednej linii
\usepackage[fit]{truncate}
% fancyhdr - ładne nagłówki, definicja wyglądu nagłówka, numery stron będą umieszczane w nagłówku po odpowiedniej stronie
\usepackage{fancyhdr}
\pagestyle{fancy}
\renewcommand{\chaptermark}[1]{\markboth{#1}{}}
\renewcommand{\sectionmark}[1]{\markright{\thesection\ #1}}
\fancyhf{}
\fancyhead[LE,RO]{\bfseries\thepage}
% tutaj ograniczamy szerokość pola w nagłówku zawierającego nazwę rozdziału/podrozdziału do 95% szerokości strony
% redefinicja sposobu prezentacji nazw domyślnie wypisywanych wielkimi literami (np. domyślnie w nagłówku Spis treści będzie miał postać SPIS TREŚCI)
% Uwaga! to może popsuć wielkie litery w ogóle! Jak coś nie działa należy usunąć \nouppercase{} z poniższych definicji
\fancyhead[LO]{\nouppercase{\bfseries{\truncate{.95\headwidth}{\rightmark}}}}
\fancyhead[RE]{\nouppercase{\bfseries{\truncate{.95\headwidth}{\leftmark}}}}
\renewcommand{\headrulewidth}{0.5pt}
\renewcommand{\footrulewidth}{0pt}

% definicja typu prostego wymagana przez pierwsze strony rozdziałów itp.
% powyższe reguły niestety tych stron nie dotyczą, gdyż Latex automatycznie przełącza je pomiędzy fancy a plain
% w tym wypadku eliminujemy nagłówki i stopki na stronach początkowych
\fancypagestyle{plain}{%
 \fancyhead{}
 \fancyfoot{}
 \renewcommand{\headrulewidth}{0pt}
 \renewcommand{\footrulewidth}{0pt}
}

\parskip 0.05in


% makro umożliwiające otaczanie symboli okręgami
\usepackage{tikz}
% brak justowania tekstu (bazą okręgu będzie linia tekstu)
\newcommand*\mycirc[1]{%
  \begin{tikzpicture}
    \node[draw,circle,inner sep=1pt] {#1};
  \end{tikzpicture}}

% pionowe justowanie tekstu, środek okręgu pokrywa się ze środkiem tekstu
\newcommand*\mycircalign[1]{%
  \begin{tikzpicture}[baseline=(C.base)]
    \node[draw,circle,inner sep=1pt](C) {#1};
  \end{tikzpicture}}

% zmiana nazwy twierdzeń i lematów
\newtheorem{theorem}{Twierdzenie}
\newtheorem{definition}{Definicja}
\newtheorem{example}{Przykład}

% tworzenie definicji dowodu
\newenvironment{proof}[1][Dowód]{\begin{trivlist}
\item[\hskip \labelsep {\bfseries #1}]}{\end{trivlist}}
% \newenvironment{definition}[1][Definicja]{\begin{trivlist}
% \item[\hskip \labelsep {\bfseries #1}]}{\end{trivlist}}
% \newenvironment{example}[1][Przykład]{\begin{trivlist}
% \item[\hskip \labelsep {\bfseries #1}]}{\end{trivlist}}
% \newenvironment{remark}[1][Uwaga]{\begin{trivlist}
% \item[\hskip \labelsep {\bfseries #1}]}{\end{trivlist}}

% definicja czarnego prostokąta zwyczajowo dodawanego na koniec dowodu
\newcommand{\qed}{\nobreak \ifvmode \relax \else
      \ifdim\lastskip<1.5em \hskip-\lastskip
      \hskip1.5em plus0em minus0.5em \fi \nobreak
      \vrule height0.75em width0.5em depth0.25em\fi}

% poniższymi instrukcjami można sterować co ma być numerowane a co nie i co ma być wyświetlane w spisie treści
% \setcounter{secnumdepth}{3}
% \setcounter{tocdepth}{5}

% definicja czcionki mniejszej niż tiny (domyślnie takiej małej nie ma)
\usepackage{lmodern}
\makeatletter
  \newcommand\tinyv{\@setfontsize\tinyv{4pt}{6}}
\makeatother

% definicja jeszcze mniejszej czcionki
\usepackage{lmodern}
\makeatletter
  \newcommand\tinyvv{\@setfontsize\tinyvv{3.5pt}{6}}
\makeatother

% pakiet do obsługi wieloicowych tabel
\usepackage{longtable}
\setlength{\LTcapwidth}{\textwidth}

\usepackage[section] {placeins}

\usepackage{multirow}

\usepackage{slantsc}

% nazwa pliku ze stroną tytułową
% % allows useg @ as a @ not as special character
% required for macro redefinition
\makeatletter

% parameters definition
% they cannot conflict with other
% like bibteh attributes etc.
\def\promotor#1{\def\@promotor{#1}}
\def\promotordpt#1{\def\@promotordept{#1}}
\def\promotoruniv#1{\def\@promotoruniv{#1}}
\def\miasto#1{\def\@miasto{#1}}

\def\maketitle{
  %removal of header
  \thispagestyle{empty}%
  %changing margins to match department requirements
%   \changepage{+60pt}{+70pt}{}{-30pt}{}{-30pt}{-12pt}{-18pt}{}
  \begin{center}
    \begin{tabular}{@{}lcr@{}}
      \multirow{4}{*}{\includegraphics[height=1.9cm]{img/logo_pg.png}} &
      \LARGE{\textbf{POLITECHNIKA GDAŃSKA}} &
      \multirow{4}{*}{\includegraphics[height=1.9cm]{img/logo_eti.png}}\\
      & & \\
      &\LARGE{\textbf{Wydział Elektroniki,}}&\\
      &\LARGE{\textbf{Telekomunikacji i Informatyki}}&
    \end{tabular}
  \end{center}
  \rule{\linewidth}{0.1mm}
  \vspace{2.82cm}
  \begin{center}
    \huge{\textbf{\@author}}
  \end{center}
  \vspace{0.5cm}
  \begin{center}
    \Huge{\textbf{\@title}}
  \end{center}
  \vspace{0.5cm}
  \begin{center}
    \huge{Rozprawa doktorska}
  \end{center}
  \vspace{1.41cm}
% \renewcommand{\arraystretch}{0.85}
  \begin{tabular}{@{}p{0.46\textwidth}@{}@{}p{0.54\textwidth}@{}}
\noalign{\smallskip}
    & \hspace{-20pt}\LARGE{Promotor:}\\
\noalign{\smallskip}
\noalign{\smallskip}
    & \Large{\@promotor}\\
\noalign{\smallskip}
\noalign{\smallskip}
    & \Large{\@promotordept}\\
\noalign{\smallskip}
\noalign{\smallskip}
    & \Large{\@promotoruniv}
  \end{tabular}
  \vfill
  \begin{center}
    \Large{\@miasto, \@date}
  \end{center}
  %
%   \cleardoublepage
  % restore page defaults
%   \changepage{-60pt}{-70pt}{}{+30pt}{}{+30pt}{+12pt}{+18pt}{}
}

%restore @ sign
\makeatother

\cleardoublepage
% allows useg @ as a @ not as special character
% required for macro redefinition
\makeatletter

% parameters definition
% they cannot conflict with other
% like bibteh attributes etc.
\def\promotor#1{\def\@promotor{#1}}
\def\miasto#1{\def\@miasto{#1}}
\def\studies#1{\def\@studies{#1}}
\def\descr#1{\def\@descr{#1}}
\def\indeks#1{\def\@indeks{#1}}
\def\dept#1{\def\@dept{#1}}

\def\maketitle{
  %removal of header
  \thispagestyle{empty}%

  \begin{center}
    \begin{tabular}{lcl}
      \multirow{4}{*}{\includegraphics[height=2.5cm]{img/logo_pg.png}} &
      \textsc{\textbf{Politechnika Gdańska}} &
      \multirow{4}{*}{\includegraphics[height=2.5cm]{img/logo_eti.png}}\\
      & & \\
      &\textsc{\textbf{Wydział Elektroniki,}}&\\
      &\textsc{\textbf{Telekomunikacji i Informatyki}}&
    \end{tabular}
  \end{center}
  \vspace{1cm}
  \begin{tabular*}{\textwidth}{p{0.5\textwidth}p{0.5\textwidth}}
    \textbf{Katedra:} & \@dept\\
    &\\
    \textbf{Imię i nazwisko dyplomanta:} & \@author\\
    &\\
    \textbf{Nr albumu:} & \@indeks\\
    &\\
    \textbf{Forma i poziom studiów:} & \@studies\\
    &\\
    \textbf{Kierunek studiów:} & Informatyka\\
  \end{tabular*}
  \begin{center}
    \vspace{1cm}
    \Large{\textbf{Praca dyplomowa magisterska}}
  \end{center}
  \vspace{1cm}
  \begin{tabular*}{\textwidth}{p{\textwidth}}
    \textbf{Temat pracy:} \\ \@title\\
    \\
    \textbf{Kierujący pracą:} \\ \@promotor\\
    \\
    \textbf{Zakres pracy:} \\ \@descr\\
  \end{tabular*}
  \vspace*{\stretch{6}}
  \begin{center}
    \@miasto, \@date
  \end{center}

}

%restore @ sign
\makeatother

\cleardoublepage

% parametry strony tytułowej, zdefiniowane są w plikach z poszczególnymi stronami
% tytuł pracy
\title{Obliczanie zer wielomianów}
% autor
\author{Wojciech Pasternak}
% rok wydania
\date{2016}
% miasto, gdzie napisano pracę
\miasto{Gdańsk}
% promotor
\promotor{dr hab. inż. Robert Janczewski}
% wydział promotora, tylko dla phd_titlepage
% \promotordpt{Wydział Elektroniki, Telekomunikacji i~Informatyki}
% uczelnia promotora, tylko dla phd_titlepage
% \promotoruniv{Politechnika Gdańska}

% rodzaj studiów, tylko dla mgr_titlepage
\studies{Stacjonarne jednolite studia magisterskie}
% opis pracy, tylko dla mgr_titlepage
\descr{Opis pracy (jedno zdanie)}
% nr indeksu, tylko dla mgr_titlepage
\indeks{137361}
% katedra, tylko dla mgr_titlepage
\dept{Architektury Systemów Komputerowych}

% korekta marginesów - domyślnie latex ma jakieś kosmiczne
\usepackage{anysize}
\marginsize{3.5cm}{2.5cm}{2.5cm}{2.5cm}
% po zmianie marginesów konieczne jest wymuszenie przeliczenia nagłówków
\fancyhfoffset[E,O]{0pt}

\begin{document}
% sekcja wstępna książki, numerowana rzymskimi
\frontmatter
% generacja strony tytułowej załączonej wcześniej
\maketitle
% spis treści
\tableofcontents

% właściwa część książki, numerowana arabskimi od 1
\mainmatter

\chapter{Przegląd literatury}
\section{Wielomiany}
\subsection{Definicja}




\begin{definition}
	Wielomianem zmiennej rzeczywistej x nazywamy wyrażenie:
	\begin{equation}
		\begin{split}
			&W(x) = a_0x^n + a_1x^{n-1} + a_2x^{n-2}+ ... + a_{n-1}x + a_n, \\
			&gdzie\ a_0, a_1, a_2, ..., a_{n-1}, a_n\in R, n \in N 
		\end{split}
	\end{equation}
\end{definition}

Liczby $a_0, a_1, a_2, ..., 1_{n-1}, a_n$ nazywamy współczynnikami wielomianu, natomiast n nazywamy stopniem wielomianu.
	
Szczególnym przypadkiem wielomianu jest jednomian. 

\begin{definition}
	Jednomianem zmiennej rzeczywistej x nazywamy wielomian, który posiada co najwyżej jeden wyraz niezerowy i określamy wzorem:
	\begin{equation}
		\begin{split}
			W(x) = ax^n, gdzie\ a\in R, n \in N 
		\end{split}
	\end{equation}
\end{definition}

Można, więc rozumieć wielomian jako skończoną sumę jednomianów.
Jednomian stopnia zerowego jest stała, pojedyncza liczba rzeczywista, która w szczególności może być zerem.

\begin{definition}
	Wielomianem zerowym nazywamy, wielomian wyrażony wzorem:
	\begin{equation}
		W(x) = 0
	\end{equation}
\end{definition}

W dalszej części, jeżeli nie zaznaczymy inaczej, mówiąc wielomian, będziemy mieli na myśli pewien wielomian, nie będący wielomianem zerowym.

\subsection{Podstawowe działania na wielomianach}

Na wielomianach, tak jak na liczbach możemy wykonywać podstawowe działania. Należą do nich: porównywanie, dodawanie, odejmowanie, mnożenie, dzielenie, a także obliczanie reszty z dzielenia oraz NWD (największego wspólnego dzielnika). Jako, że wielomian zmiennej x możemy traktować jak funkcję jednej zmiennej, możemy także policzyć z niego pochodne.

\subsubsection{Porównywania wielomianów}

\begin{theorem}
	Dwa wielomiany uważamy za równe wtedy i tylko wtedy, gdy są tego samego stopnia, a ich kolejne współczynniki są równe.
\end{theorem}

\begin{example}
	Mamy dane wielomian $W_1$ oraz wielomian $W_2$.
	\begin{equation}
		\begin{split}
			&W_1(x) = a_0x^n + a_1x^{n-1} + ... + a_{n-1}x + a_n \\
			&W_2(x) = b_0x^n + b_1x^{n-1} + ... + b_{n-1}x + b_n \\
		\end{split}
	\end{equation}
	Wielomiany $W_1$ oraz $W_2 $ są równe wtedy i tylko wtedy gdy
	$\forall{i\in N}\ a_i = b_i$.
\end{example}

\subsubsection{Suma wielomianów}

\begin{theorem}
	Aby dodać dwa wielomiany, należy dodać ich wyrazy podobne.
\end{theorem}
\begin{example}
	Mamy dane wielomian $W_1$ oraz wielomian $W_2$.
	\begin{equation}
		\begin{split}
			&W_1(x) = a_0x^n + a_1x^{n-1} + ... + a_{n-1}x + a_n \\
			&W_2(x) = b_0x^n + b_1x^{n-1} + ... + b_{n-1}x + b_n \\
		\end{split}
	\end{equation}
	Zdefiniujmy trzeci wielomian: $W_3(x) = W_1(x) + W_2(x)$. Wówczas:
	\begin{equation}
		W_3(x) = (a_0+b_0)x^n + (a_1+b_1)x^{n-1} + ... + (a_{n-1} + b_{n-1})x + a_n + b_n
	\end{equation}
\end{example}

Na powyższym przykładzie łatwo zaobserwować, że stopień sumy dwóch wielomianów nie może być większy od większego ze stopni dodawanych wielomianów. W przypadku gdy oba te wielomiany są tego samego stopnia, o przeciwnym współczynniku przy najwyższej potędze, to stopień ten będzie mniejszy.

\begin{theorem}
	\begin{equation}
		\deg(W_1 + W_2) \le max(deg(W_1),\ deg (W_2))
	\end{equation}
\end{theorem}

\subsubsection{Różnica wielomianów}

\begin{definition}
	Wielomianem przeciwnym nazywamy wielomian, którego wszystkie współczynniki są przeciwne do danych.
\end{definition}

\begin{example}
	Mamy dany wielomian $W_1$.
	\begin{equation}
		W_1(x) = a_0x^n + a_1x^{n-1} + ... + a_{n-1}x + a_n
	\end{equation}
	Zdefiniujmy drugi wielomian: $W_2(x) = -W_1(x)$. Wówczas:
	\begin{equation}
		W_1(x) = -a_0x^n + (-a_1)x^{n-1} + ... + (-a_{n-1})x + (-a_n)
	\end{equation}
\end{example}

\begin{theorem}
	Aby obliczyć różnicę wielomianów $W_1$ i $W_2$, należy dodać ze sobą wielomiany $W_1$ i $-W_2$, czyli wielomian przeciwny do wielomianu $W_2$
\end{theorem}

\begin{example}
	Mamy dane wielomian $W_1$ oraz wielomian $W_2$.
	\begin{equation}
	\begin{split}
	&W_1(x) = a_0x^n + a_1x^{n-1} + ... + a_{n-1}x + a_n \\
	&W_2(x) = b_0x^n + b_1x^{n-1} + ... + b_{n-1}x + b_n \\
	\end{split}
	\end{equation}
	Zdefiniujmy trzeci wielomian: $W_3(x) = W_1(x) - W_2(x)$. Wówczas:
	\begin{equation}
		W_3(x) = (a_0-b_0)x^n + (a_1-b_1)x^{n-1} + ... + (a_{n-1} - b_{n-1})x + a_n - b_n
	\end{equation}
\end{example}

Wielomianem neutralnym ze względu na dodawanie i odejmowanie jest wielomian $W(x)=0$.

\subsubsection{Iloczyn wielomianów}

\begin{theorem}
	Aby pomnożyć dwa wielomiany, należy wymnożyć przez siebie wyrazów obu wielomianów, a następnie dodać do siebie wyrazy podobne.
\end{theorem}

\begin{example}
	Mamy dane wielomian $W_1$ oraz wielomian $W_2$.
	\begin{equation}
	\begin{split}
	&W_1(x) = a_0x^n + a_1x^{n-1} + ... + a_{n-1}x + a_n \\
	&W_2(x) = b_0x^m + b_1x^{m-1} + ... + b_{m-1}x + b_m \\
	\end{split}
	\end{equation}
	Zdefiniujmy trzeci wielomian: $W_3(x) = W_1(x) * W_2(x)$. Wówczas:
	\begin{equation}
	W_3(x) = (a_0*b_0)x^{n+m} + (a_0b_1+a_1b_0)x^{n+m-1} + ... + (a_{n-1}b_m + a_nb_{m-1})x + a_nb_m
	\end{equation}
\end{example}

Na podstawie powyższego przykładu, możemy zaobserwować, że stopień wielomianu, będącego iloczynem dwóch wielomianów niezerowych, jest równy sumie stopni tych wielomianów. Jeżeli jeden z czynników jest wielomianem zerowym, to stopień iloczynu jest równy 0.

\begin{theorem}
	\begin{equation}
	\begin{split}
	&deg(W_1 * W_2) = deg(W_1) + deg(W_2),\ dla\ W_1 != 0, W_2 != 0 \\
	&deg(W_1 * W_2) = 0,\ w \ pozostałych\ przypadkach
	\end{split}
	\end{equation}
\end{theorem}

\subsubsection{Iloraz wielomianów}

\begin{definition}
	Wielomian W(x) nazywamy podzielnym przez niezerowy wielomian P(x) wtedy i tylko wtedy, gdy istnieje taki wielomian Q(x), że spełniony jest warunek $W(x) = P(x) * Q(x)$. Wówczas: wielomian Q(x) nazywamy ilorazem wielomianu W(x) przez P(x), zaś wielomian P(x) nazywamy dzielnikiem wielomianu W(x).
\end{definition}

\begin{definition}
	Wielomian R(x) nazywamy resztą z dzielenia wielomianu W(x) przez niezerowy wielomian P(x) wtedy i tylko wtedy, gdy istnieje taki wielomian Q(x), że spełniony jest warunek $W(x) = P(x) * Q(x) + R(x)$.
\end{definition}

\begin{example}
	Mamy dane wielomian W oraz wielomian P.
	\begin{equation}
		\begin{split}
			&W(x) = a_0x^n + a_1x^{n-1} + ... + a_{n-1}x + a_n \\
			&P(x) = b_0x^m + b_1x^{m-1} + ... + b_{m-1}x + b_m \\
		\end{split}
	\end{equation}
	Zdefiniujmy wielomian: $Q(x) = W(x) / P(x)$ oraz $Q(x) = W(x)\ mod\ P(x)$. Wówczas:
	\begin{equation}
		\begin{split}
			&Q(x) = c_0x^{n-m} + c_1x^{n-m-1} + ... + c_{n-m-1}x + c_{n-m} \\
			&R(x) = d_0x^{n-m-1} + d_1x^{n-m-2} + ... + d_{n-m-2}x + d_{n-m-1} \\
		\end{split}
	\end{equation}
\end{example}

\begin{theorem}
	\begin{equation}
	deg(W_1\ mod\ W_2) < deg(W_1 / W_2) = deg(W_1) - deg(W_2)
	\end{equation}
\end{theorem}


\subsubsection{Pochodna wielomianu}

\begin{definition}
	Dany jest wielomian W, określony wzorem $W(x) = a_0x^n + a_1x^{n-1} + ... + a_{n-1}x + a_n$. Pochodną wielomianu W nazywamy wielomian W' i wyrażamy wzorem:
	\begin{equation}
	W(x) = na_0x^{n-1} + (n-1)a_1x^{n-2} + ... + 2a_{n-2}x + a_{n-1}
	\end{equation}
\end{definition}

\begin{theorem}
	\begin{equation}
	\begin{split}
		&deg(W') = deg(W) - 1,\ dla\ deg(W) > 0 \\
		&W = 0,\ w\ pozostałych\ przypadkach
	\end{split}
	\end{equation}
\end{theorem}

\subsubsection{NWD wielomianów}




\begin{theorem}
	Jeżeli wielomian W(x) podzelimy przez dwumian $x - x_0$, to reszta z tego dzielenia jest równa wartości tego wielomianu dla $x = x_0$.
\end{theorem}

\begin{theorem}[Bezout]
	Liczba $x_0$ jest pierwiastkiem wielomianu W(x) wtedy i tylko wtedy, gdy wielomian jest podzielny przez dwumian $x - x_0$
\end{theorem}

\begin{theorem}
	Liczba $x_0$ jest pierwiastkiem k-krotnym wielomianu W(x) wtedy i tylko wtedy, gdy wielomian jest podzielny przez $(x - x_0)^k$ i nie jest podzielny przez $(x - x_0)^{k+1}$.
\end{theorem}

\begin{theorem}
	Każdy wielomian W(x) nie będący wielomianem zerowym jest iloczynem czynników stopnia co najwyżej drugiego
\end{theorem}

\begin{theorem}
	Niezerowy wielomian, o współczynnikach rzeczywistych, jest jednoznacznie rozkładalny na czynniki liniowe lub nierozkładalne czynniki kwadratowe, o współczynnikach rzeczywistych.
\end{theorem}

\begin{theorem}
	Dany jest wielomian $W(x) = x^n + a_{n-1}x^{n-1} + ... + a_1x + a_0$, o współczynnikach całkowitych. Jeżeli wielomian W posiada pierwiastki całkowite, to są one dzielnikami wyrazu wolnego $a_0$.
\end{theorem}

\begin{theorem}
	Dowolny wielomian $W_1(x) = \frac{a_n}{b_n}x^n + \frac{a_{n-1}}{b_{n-1}}x^{n-1} + ... + \frac{a_1}{b_1}x + \frac{a_0}{b_0}$, o współczynnikach wymiernych, można przekształcić w wielomian $W_2(x) = k * W_1(x)$, o współczynnikach całkowitych i tych samych pierwiastkach, co wielomian W. Wówczas:
	\begin{equation}
	k = NWW(b_0, b_1, ..., b_{n-1}, b_n)
	\end{equation}
\end{theorem}

\begin{theorem}
	Jeżeli liczba jest pierwiastkiem k-krotnym wielomianu W, to jest pierwiastkiem (k-1)-krotnym pochodnej tego wielomianu.
\end{theorem}

.\\
.\\
.\\
.\\
.\\
.\\\\
.\\\\
.\\\\
.\\
.\\
.\\
.\\
.\\
.\\\\
.\\\\
.\\\\
.\\
.\\
.\\
.\\
.\\
.\\\\
.\\\\
.\\\\






\end{document}